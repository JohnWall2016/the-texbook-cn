% -*- coding: utf-8 -*-

\input macros

%\beginchapter Chapter 25. Summary of\\Horizontal Mode
\beginchapter Chapter 25. 水平模式总结

\origpageno=285

%^^{horizontal mode}
%Continuing the survey that was begun in Chapter 24, let us investigate
%exactly what \TeX's digestive processes can do, when \TeX\ is building
%lists in horizontal mode or in restricted horizontal mode.
\1我们继续从第 24 章开始的总结;现在讨论 \TeX\
在水平模式和受限水平模式下构建列表时,它的消化过程可以做的事情。

%\ninepoint
%\def\\{\smallbreak\textindent{$\bull$}}
%\medbreak
%\centerline{$*\qquad*\qquad*$}
%\medskip\noindent
%Three asterisks, just like those that appear here, can be found near the
%end of Chapter~24.
%Everything preceding the three asterisks in that chapter applies to
%horizontal mode as well as to vertical mode, so we need not repeat
%all those rules. In particular, Chapter~24 explains assignment commands,
%and it tells how kerns, penalties, marks, insertions, adjustments,
%and ``whatsits'' are put into horizontal lists. Our present goal
%is to consider the commands that have an intrinsically horizontal
%flavor, in the sense that they behave differently in horizontal
%mode than they do in vertical or math modes.
\ninepoint
\def\\{\smallbreak\textindent{$\bull$}}
\medbreak
\centerline{$*\qquad*\qquad*$}
\medskip\noindent
类似这样的三个星号在第 24 章快结束时可以找到。
在第 24 章中,在三个星号前面的所有内容都既适用于水平模式,也适用于垂直模式,
因此我们就毋需重复那些规则了。
特别地,第 24 章讨论了赋值命令,并且讨论了紧排、惩罚、标记、插入项、
调整和``无名''是如何放入水平列中的。
我们当前的目标是讨论本质上属于水平模式的命令,
即在水平模式下与在垂直和数学模式下表现不同的命令。

%One of the things characteristic of horizontal mode is the ``^{space
%factor},'' which modifies the width of spaces as described in Chapter~12.
%If a command changes the value of\/ ^|\spacefactor|, that fact is
%specifically noted here. The space factor is initially set to~1000, when
%\TeX\ begins to form a horizontal list, except in the case of\/ |\valign| and
%|\noalign| when the space factor of the outer list continues inside the
%inner one.
水平模式一个特有的东西是``间距因子'', 它象第十二章讨论的那样来修改间距的宽度。%
如果要用命令改变 |\spacefactor| 的值,那么就要特别注意一下。%
当 \TeX\ 开始形成水平列时,间距因子最初设定为 1000,
但是在 |\valign| 和 |\noalign| 时,外层列的间距因子会延续到内层来。

%\\^|\hskip|\<glue>, ^|\hfil|, ^|\hfill|, ^|\hss|, ^|\hfilneg|.\enskip
%A glue item is appended to the current horizontal list.
\\^|\hskip|\<glue>, ^|\hfil|, ^|\hfill|, ^|\hss|, ^|\hfilneg|.\enskip
把一个粘连项目追加到当前列。

%\\\<leaders>\<box or rule>\<horizontal skip>.\enskip
%Here ^\<horizontal skip> refers to one of the five glue-appending commands
%just mentioned; the formal syntax for \<leaders> and for \<box or rule> is
%given in Chapter~24. A glue item that produces ^{leaders} is appended.
\\\<leaders>\<box or rule>\<horizontal skip>.\enskip
这里的 \<horizontal skip> 指的是刚刚讨论的五个追加粘连的命令;
\<leaders> 和 \<box or rule> 的正式语法见第二十四章。%
生成指引线的粘连项目要追加到当前列。

%\\^\<space token>.\enskip
%Spaces append glue to the current list; the exact amount of glue depends on
%|\spacefactor|, the current font, and the |\spaceskip| and |\xspaceskip|
%parameters, as described in Chapter~12.
\\^\<space token>.\enskip
空白把粘连追加到当前列;
粘连的具体量与 |\spacefactor|, 当前字体,|\spaceskip| 和 |\xspaceskip| 有关,
见第十二章的讨论。

%\\|\|\].\enskip ^^{control space}
%A control-space command appends glue to the current list, using the same amount
%that a \<space token> inserts when the space factor is 1000.
\\|\|\].\enskip ^^{control space}
控制空格命令把粘连追加到当前列,
粘连的大小与间距因子为 1000 时 \<space token> 所插入粘连的大小相同。

%\\^\<box>.\enskip
%The box is constructed, and if the result is void nothing happens.
%Otherwise the new box is appended to the current list, and
%the space factor is~set~to~1000.
\\^\<box>.\enskip
构造这个盒子,并且如果所得到的是置空的就什么也不会出现。%
否则,把新盒子追加到当前列,并且把间距因子设置为 1000。

%\\^|\raise|\<dimen>\<box>, ^|\lower|\<dimen>\<box>.\enskip
%This acts just like an unadorned \<box> command, except that the new box
%being appended to the horizontal list is also shifted up or down by the
%specified amount.
\\^|\raise|\<dimen>\<box>, ^|\lower|\<dimen>\<box>.\enskip
它与普通的 \<box> 命令是一样的,
但是追加到水平列的新盒子还要向上或向下平移所给定的量。

%\\^|\unhbox|\<8-bit number>, ^|\unhcopy|\<8-bit number>.\enskip
%If the specified box register is void, nothing happens. Otherwise that
%register must contain an hbox. The horizontal list inside that box is
%appended to the current horizontal list, without changing it in any way.
%The value of\/ |\spacefactor| is not affected. The box register becomes void
%after |\unhbox|, but it remains unchanged by |\unhcopy|.
\\^|\unhbox|\<8-bit number>, ^|\unhcopy|\<8-bit number>.\enskip
如果给定的盒子寄存器是置空的,那么什么也不出现。%
否则,寄存器必须包含一个 hbox。%
在此盒子中的水平列要追加到当前水平列,而不做任何改变。%
|\spacefactor| 的值不受影响。%
在 |\unhbox| 后此盒子寄存器变成置空的,在 |\unhcopy| 后它保持不变。

%\\\<vertical rule>.\enskip
%The specified ^{rule} is appended to the current list, and the |\spacefactor|
%is set to 1000.
\\\<vertical rule>.\enskip
把给定的标尺追加到当前列,并且把 |\spacefactor| 设置为 1000。

%\\^|\valign|\<box specification>|{|\<alignment material>|}|.\enskip
%The ^\<alignment material> consists of a preamble followed by zero or more
%columns to be aligned; see Chapter~22. \TeX\ enters a new level of grouping,
%represented by the `|{|' and `|}|', within which changes to ^|\tabskip|
%will be confined.  The alignment material can also contain optional
%occurrences of `|\noalign|\<filler>|{|\<horizontal mode material>|}|'
%between columns; this adds another level of grouping.  \TeX\ operates in
%restricted horizontal mode while it works on the material in ^|\noalign|
%\vadjust{\eject}% squeeze another line onto page 285, we need it on page 286!
%groups and when it appends columns of the \hbox{alignment}; the resulting
%internal horizontal list will be appended to the enclosing horizontal list
%after the alignment is completed. The value of\/ |\spacefactor| at the time
%of the |\valign| is used at the beginning of the internal horizontal list,
%and the final value of\/ |\spacefactor| is carried to the enclosing
%horizontal list when the alignment is completed.  The space factor is set
%to 1000 after each column; hence it affects the results only in |\noalign|
%groups.  \TeX\ also enters an additional level of grouping when it works
%on each individual entry of the alignment, during which time it acts in
%internal vertical mode; the individual entries will be vboxed as part of
%the final alignment.
%%No room for the following redundant remarks:
%%  The commands |\noalign|, |\omit|, |\span|, |\cr|,
%%|\crcr|, and |&| (where |&| denotes an explicit or implicit character of
%%category~4) are intercepted by the alignment process, en route to \TeX's
%%stomach, so they will not appear as commands in the stomach unless \TeX\
%%has lost track of what alignment they belong to.
\\^|\valign|\<box specification>|{|\<alignment material>|}|.\enskip
\<alignment material> 由导言后面跟零个或更多个要对齐的栏组成;
见第二十二章。%
 \TeX\ 进入由`|{|'和`|}|'表示的一个新层次的编组,
其中对 |\tabskip| 的修改是受到限制的。%
对齐的内容还可以包含栏之间的`|\noalign|\<filler>|{|\<horizontal mode material>|}|';
这增加了另一个层次的编组。%
当处理 |\noalign| 组中的内容并且当追加对齐的栏时, \TeX\ 处于受限水平模式中;
\1在对齐结束后,所得到的不变水平列将追加到封装的水平列。%
在 |\valign| 中 |\spacefactor| 的值使用的是内部水平列开始时的值,
当对齐结束后,~|\spacefactor| 最后的值要送到封装的水平列。%
在每栏结束后,间距因子就设置为 1000;
因此它只影响在 |\noalign| 组中的结果。%
当处理对齐的每个栏时, \TeX\ 还要进入更深层的编组,
此时它处在内部垂直模式;
各个单元都是作为最后对齐的一部分而放在 vbox 中。

%\\^|\indent|.\enskip
%An empty box of width ^|\parindent| is appended to the current list, and
%the space factor is set to 1000.
\\^|\indent|.\enskip
把宽度为 |\parindent| 的空盒子追加到当前列,
并且把间距因子设置为 1000。

%\\^|\noindent|.\enskip
%This command has no effect in horizontal modes.
\\^|\noindent|.\enskip
此命令在水平模式下没有什么作用。

%\\^|\par|.\enskip
%The primitive |\par| command, also called ^|\endgraf| in plain \TeX,
%does nothing in restricted horizontal mode. But it terminates horizontal
%mode: The current list is finished off by doing ^|\unskip| ^|\penalty10000|
%^^|\parfillskip|
%|\hskip\parfillskip|, then it is broken into lines as explained in Chapter~14,
%and \TeX\ returns to the enclosing vertical or internal vertical mode.
%The lines of the paragraph are appended to the enclosing vertical list,
%interspersed with interline glue and interline penalties, and with
%the ^{migration} of vertical material that was in the horizontal list.
%Then \TeX\ exercises the page builder.
\\^|\par|.\enskip
原始命令 |\par|, 在 plain \TeX\ 中也称为 |\endgraf|,
它在受限水平模式下什么也不做。%
但是它终止当前水平模式:当前列通过
命令 |\unskip| |\penalty10000| |\hskip\parfillskip| 来完成,
接着,它如第十四章讨论的那样分段为行,
并且 \TeX\ 回到封装的垂直或内部垂直模式。%
把段落的行追加到封装的垂直列,其间插入行间粘连和行间惩罚,并且把水平列中%
的垂直内容转移出来。%
接着 \TeX\ 进行页面构建。

%\\|{|.\enskip
%A character token of category 1, or a control sequence like~|\bgroup|
%that has been |\let| equal to such a character token, causes \TeX\ to
%start a new level of ^{grouping}. When such a group ends---with `|}|'---\TeX\
%will undo the effects of non-global assignments without leaving whatever
%mode it is in at that time.
\\|{|.\enskip
类别码为 1 的字符记号,或者用 |\let| 使得它与这种字符记号相等的控制序列,
比如 |\bgroup|,将让 \TeX\ 开始一个新层级的^{编组}。
当这样的编组用 `|}|' 结束后,\TeX\ 将撤消所有的非全局赋值,并且保持其目前的模式不变。

%\\Some commands are incompatible with horizontal mode because they are
%intrinsically vertical. When the following commands appear in unrestricted
%horizontal mode, they cause \TeX\ to conclude the current paragraph:
%^^{paragraph end, implied}
%\beginsyntax
%<vertical command>\is^|\unvbox|\alt^|\unvcopy|\alt^|\halign|\alt^|\hrule|
%  \alt^|\vskip|\alt^|\vfil|\alt^|\vfill|\alt^|\vss|\alt^|\vfilneg|%
%  \alt^|\end|\alt^|\dump|
%\endsyntax
%The appearance of a \<vertical command> in restricted horizontal mode is
%forbidden, but in regular horizontal mode it causes \TeX\ to insert the
%token \cstok{par} into the input; after reading and expanding this \cstok{par}
%token, \TeX\ will see the \<vertical command> token again. \ (The current
%meaning of the control sequence ^|\par| will be used; \cstok{par} might no
%longer stand for \TeX's |\par| primitive.)
\\某些命令与水平模式不相容,因为它们在本质上是垂直模式的。%
当下列命令出现在非受限水平模式中时,会使 \TeX\ 结束当前段落:
\beginsyntax
<vertical command>\is^|\unvbox|\alt^|\unvcopy|\alt^|\halign|\alt^|\hrule|
  \alt^|\vskip|\alt^|\vfil|\alt^|\vfill|\alt^|\vss|\alt^|\vfilneg|%
  \alt^|\end|\alt^|\dump|
\endsyntax
在受限水平模式下,不允许出现一个 \<vertical command>,
但是在正常水平模式下,这会使得 \TeX\ 把记号 \cstok{par} 插入到输入中;
在读入和展开这个 \cstok{par} 后, \TeX\ 就再次遇见了 \<vertical command> 记号。%
(要用到控制系列 |\par| 当前的意思;
\cstok{par} 可能不再表示 \TeX\ 的原始命令 |\par|。)

%\\\<letter>, \<otherchar>, \kern-1pt^|\char|\<8-bit number>, \<chardef token>,
%\kern-1pt^|\noboundary|.\enskip
%The most common commands of all are the character commands that tell
%\TeX\ to append a character to the current horizontal
%list, using the current font.
%If two or more commands of this type occur in succession, \TeX\ processes
%them all as a unit, converting to ligatures and/or
%inserting kerns as directed by the font information. \ (Ligatures and
%kerns may be influenced by invisible ``boundary'' characters at the left
%and right, unless |\noboundary| appears.) \ Each character
%command adjusts ^|\spacefactor|, using
%the ^|\sfcode| table as described in Chapter~12.
%In unrestricted horizontal mode, a
%`|\discretionary{}{}{}|' item is appended after a character whose code is
%the ^|\hyphenchar| of its font, or after a ligature formed from a sequence
%that ends with such a character. ^^|\discretionary|
\\\<letter>, \<otherchar>, \kern-1pt^|\char|\<8-bit number>, \<chardef token>,
\kern-1pt^|\noboundary|.\enskip
所有命令中最常用的字符命令,它把一个当前字体的字符追加到当前水平列。
如果两个或多个这种命令连续出现,\TeX\ 就把它们看作一个整体,
按照字体中的信息将它们连写和/或插入紧排。(如果 |\noboundary| 不出现的话,
连写和紧排可能会受字符左右两边看不见的``边界''的影响。)
每个字符命令都用第 12 章讨论的 |\sfcode| 表调整 |\spacefactor|。
在非受限水平模式中,在当前字体的 |\hyphenchar| 的字符后面,
或者在以这个字符结尾的序列所形成的连写后面。要添加 `|\discretionary{}{}{}|' 项。

%\\^|\accent|\<8-bit number>\<optional assignments>.
%Here ^\<optional assignments> stands for zero or more \<assignment>
%commands other than ^|\setbox|.
%If the assignments are not followed by a \<character>, where
%\<character> stands for any of the commands just discussed in the previous
%paragraph, \TeX\ treats |\accent| as if it were |\char|, except that
%the space factor is set to 1000. Otherwise the character that follows
%the assignment is accented by the character that corresponds to the
%\<8-bit number>. \ (The purpose of the intervening assignments is to
%allow the accenter and accentee to be in different fonts.) \ If the
%accent must be moved up or down, it is put into an hbox that is
%raised or lowered. Then the accent is effectively superposed on the
%character by means of kerns, in such a way that the width of the accent
%does not influence the width of the resulting horizontal list.
%Finally, \TeX\ sets |\spacefactor=1000|.
\\^|\accent|\<8-bit number>\<optional assignments>.
这里的 \<optional assignments> 表示零个或多个不为 ^|\setbox| 的 \<assignment> 命令。
如果紧随赋值之后的不是一个 \<character>,
其中 \<character> 表示前一个段落刚刚讨论的任何命令,
那么 \TeX\ 把 |\accent| 当成 |\char| 来处理,但是间距因子为 1000。
\1否则,跟在赋值后面的字符就被对应于 \<8-bit number> 的字符加上重音。%
(中间可以有赋值,目的在于允许所加的重音符和被加重音的字符为不同的字体。)
如果这个重音必须向上或向下移动,那么就把它放在一个 hbox 中来升降。
这样重音就通过调整间距的方式叠放在字符上,按照这种方法,
重音的宽度不会影响所得到的水平列的宽度。
最后,\TeX\ 设置 |\spacefactor=1000|。

%\\^|\/|.\enskip
%If the last item on the current list is a character or ligature, an
%explicit kern for its ^{italic correction} is appended.
\\^|\/|.\enskip
如果当前列上的最后一个项目是字符或连写,则会追加倾斜校正所需的显式紧排。

%\\^|\discretionary|\<general text>\<general text>\<general text>.\enskip
%The three general texts are processed in restricted horizontal mode. They
%should contain only fixed-width things; hence they aren't really very
%general in this case. More precisely, the horizontal list formed by each
%discretionary general text must consist only of characters, ligatures,
%kerns, boxes, and rules; there should be no glue or penalty items, etc.
%This command appends a discretionary item to the current list; see
%Chapter~14 for the meaning of a discretionary item. The space factor is
%not changed.
\\^|\discretionary|\<general text>\<general text>\<general text>.\enskip
这三个通用文本在受限水平模式下来处理。它们只包含宽度固定的内容;
因此,在这种情况下它们不真那么通用。更准确地说,
由任意可断点的通用文本形成的水平列只能包含字符、连写、紧排、盒子和标尺;
不应该有粘连或惩罚项等。这个命令把一个任意可断点项目追加到当前列;
见第 14 章任意可断点的含义的讨论。它不改变间距因子。

%\\^|\-|.\enskip
%This ``discretionary hyphen'' command is defined in Appendix H.
\\^|\-|.\enskip
这个``任意连字符''命令在附录 H 中给出定义。

%\\^|\setlanguage|\<number>.\enskip
%See the conclusion of Appendix H.
\\^|\setlanguage|\<number>.\enskip
见附录 H 的结果。

%\\|$|.\enskip
%A ``^{math shift}'' character causes \TeX\ to enter math mode or display math
%mode in the following way:
%\TeX\ looks at the following token without expanding it. If that token
%is a~|$| and if \TeX\ is currently in unrestricted horizontal mode,
%then \TeX\ breaks the current paragraph into lines as explained above
%(unless the current list is empty), returns to the enclosing vertical
%mode or internal vertical mode, calculates values like |\prevgraf|
%and |\displaywidth| and |\predisplaysize|, enters a new level of grouping,
%inserts the |\everydisplay| tokens into the input, exercises the page
%builder, processes `\<math mode material>|$$|' in display math mode, puts
%the display into the enclosing vertical list as explained in Chapter~19
%(letting vertical material ^{migrate}), exercises the page builder again,
%increases |\prevgraf| by~3, and resumes horizontal mode again, with an
%empty list and with the space factor equal to~1000. \ (You got that?) \
%Otherwise \TeX\ puts the looked-at token back into the
%input, enters a new level of grouping, inserts the |\everymath| tokens,
%and processes `\<math mode material>|$|'; the math mode material
%is converted to a horizontal list and appended to the current list,
%surrounded by ``math-on'' and ``math-off'' items, and the space factor
%is set to~1000. One consequence of these rules is that `|$$|' in
%restricted horizontal mode simply yields an empty math formula.
\\|$|.\enskip
``数学转换''符使 \TeX\ 按下列方式进入数学模式或陈列数学模式:
 \TeX\ 遇见下列记号而不展开它们。%
如果此记号是一个 |$| 并且 \TeX\ 目前处在非受限水平模式中,
那么 \TeX\ 象上面讨论的那样把当前段落分为行(除非当前列是空的),
再返回封装的垂直模式或内部垂直模式,计算象 |\prevdepth|,
|\displaywidth| 和 |\predisplaysize| 的值,进入新一层次的编组,
把记号 |\everydisplay| 插入到输入中,进行页面构建,在陈列数学模式下处理%
`\<math mode material>|$$|', 象第十九章讨论的那样把此陈列公式放在%
封装的垂直列中(把垂直内容转移处理),
再次进行页面构建,把 |\prevgraf| 增加 3, 并且再回到水平模式,
开始一个空列并且设置间距因子为 1000。%
(你看懂了吗")
否则, \TeX\ 就把所读入的记号放回输入,进入一个新层次的编组,
插入记号 |\everymath|, 并且处理`\<math mode material>|$|';
这个数学模式的内容被转换为一个水平列并且追加到当前列,
两边放上``math-on''项目和``math-off''项目,
并且把间距因子设置为 1000。%
这些规则的一个后果就是,在受限水平模式下的`|$$|'直接得到一个空数学公式。

%\\None of the above: If any other primitive command of \TeX\ occurs
%in horizontal mode, an error message will be given, and \TeX\ will try
%to recover in a reasonable way. For example, if a superscript or
%subscript symbol appears, or if any other inherently mathematical
%command is given, \TeX\ will try to insert a `|$|' just before the
%offending token; this will enter math mode.
%%No room for the following redundant remarks:
%%On the other hand if a totally
%%misplaced token like |\endcsname| or |\omit| or |\eqno| or |#| appears
%%in horizontal mode, \TeX\ will simply ignore it, after reporting
%%the error. You might enjoy trying to type some really stupid input,
%%just to see what happens. \ (Say `|\tracingall|' first, as explained
%%in Chapter~27, in order to get maximum information.)
\\除上面的以外:如果任何其它 \TeX\ 的原始命令出现在水平模式中,
那么就给出错误信息,
并且 \TeX\ 试着按照合理的方式修复它。%
例如,如果出现了上标或下标,或者出现了其它数学特有的命令,
那么 \TeX\ 就试着在这个讨厌的记号紧前面插入一个`|$|';
它就进入了数学模式。

\endchapter

\strut{\rm Otherwise.} %
  You may reduce all\/ \kern.5pt{\rm Verticals} into\/ {\rm Horizontals}.
\author JOSEPH ^{MOXON}, {\sl A Tutor to Astronomie and Geographie\/} (1659)

\bigskip

\strut\tt!~You can't use `\bslash moveleft' in horizontal mode.
\author \TeX\ (1982)

\vfill\eject\byebye
