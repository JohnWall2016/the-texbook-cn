% -*- coding: utf-8 -*-

\input macros

%\beginchapter Chapter 27. Recovery from\\Errors
\beginchapter Chapter 27. 错误修复

\origpageno=295

%OK, everything you need to know about \TeX\ has been explained---unless you
%happen to be fallible. If you don't plan to make any errors, don't bother to
%read this chapter. Otherwise you might find it helpful to make use of some
%of the ways that \TeX\ tries to pinpoint bugs in your manuscript.
\1好了,所有你应该知道的 \TeX\ 的东西都讨论完了——只要你不爱出错。%
如果你预计不出错,那么不用费心来看本章。%
否则,看看 \TeX\ 怎样定位你的文稿中的破绽会很有帮助。

%In the trial runs you did when reading Chapter 6, you learned the general
%form of ^{error messages}, and you also learned the various ways in which
%you can respond to \TeX's complaints. With practice, you will be able to
%correct most errors ``online,'' as soon as \TeX\ has detected them, by
%inserting and deleting a few things. The right way to go about this is to
%be in a mellow mood when you approach \TeX, and to regard the error
%messages that you get as amusing puzzles---``Why did the machine do
%that?''---rather than as personal insults.
在第六章中的实战中,你已经知道了错误信息的一般形式,
并且学会了对付 \TeX\ 找到的毛病的几种方法。%
在实践中,你可以``实时''修正大多数错误,只要 \TeX\ 检测到它们,就可以插入或%
删除一些东西。%
对待它的正确方法是在你处理 \TeX\ 时要老练,并且把错误信息看作有趣的难题——%
``为什么计算机会这样?''——而不是觉得有失体面。

%\TeX\ knows how to issue more than a hundred different sorts of error messages,
%and you probably never will encounter all of them, because some types of
%mistakes are very hard to make. We discussed the ``undefined control sequence''
%error in Chapter~6; let's take a look at a few of the others now.
 \TeX\ 能给出一百多种的错误信息,
可能你从不会遇见所有的这些,以外有些错误很难出现。%
在第六章我们讨论了错误``undefined control sequence'';
现在我们再看看其它几个。

%If you misspell the name of some unit of measure---for example, if you
%type `|\hsize=4im|' instead of `|\hsize=4in|'---you'll get an error
%message that looks something like this:
%\begintt
%! Illegal unit of measure (pt inserted).
%<to be read again>
%                   i
%<to be read again>
%                   m
%<*> \hsize=4im
%               \input story
%?
%\endtt
%^^|Illegal unit|
%\TeX\ needs to see a legal unit before it can proceed; so in this case it has
%implicitly inserted `|pt|' at the current place in the input, and it has
%set |\hsize=4pt|.
如果你拼错了某些测量单位——例如,如果你输入的是`|\hsize=4im|'而%
不是`|\hsize=4in|'——就得到象下面这样的错误信息:
\begintt
! Illegal unit of measure (pt inserted).
<to be read again>
                   i
<to be read again>
                   m
<*> \hsize=4im
               \input story
?
\endtt
在 \TeX\ 处理之前要读入应该合法的单位;
因此在这种情况下它就在输入的当前位置处暗中插入了`|pt|',
并且设置 |\hsize=4pt|。

%What's the best way to ^{recover} from such an error? Well, you should
%always type `|H|' or `|h|' to see the help message, if you aren't sure
%what the error message means. Then you can look at the lines of context
%and see that \TeX\ will read `|i|' and then `|m|' and then `~|\input|
%|story|~' if you simply hit \<return> and carry on. Unfortunately, this
%easy solution isn't very good, because the `|i|' and~`|m|' will be typeset
%as part of the text of a new paragraph.  A much more graceful recovery is
%possible in this case, by first typing~`|2|'.  This tells \TeX\ to discard
%the next two tokens that it reads; and after \TeX\ has done so, it will
%stop again in order to give you a chance to look over the new situation.
%Here is what you will see:
%\begintt
%<recently read> m
%|indent
%<*> \hsize=4im
%               \input story
%?
%\endtt
%Good; the `|i|' and `|m|' are read and gone. But if you hit \<return> now,
%\TeX\ will `|\input story|' and try to typeset the |story.tex| file with
%|\hsize=4pt|; that won't be an especially exciting experiment, because it
%will simply produce dozens of overfull boxes, one for every syllable
%of the story. Once again there's a better way: You can insert the command
%that you had originally intended, by typing
%\begintt
%I\hsize=4in
%\endtt
%now. This instructs \TeX\ to change |\hsize| to the correct value, after
%which it will |\input story| and you'll be on your way.
修复这种错误的最好方法是什么呢?
嗯,如果你不明白错误信息的意思,总可以键入`|H|'或`|h|'来查看帮助信息。%
接着你可以查看前后的行,并且看看是否你直接键入 \<return>~ \TeX\ 就依次读入`|i|'%
 `|m|'以及`|\input| |story|'并且进行下去。%
不幸的是,这种简单的方法并不很好,因为`|i|' 和 `|m|'将作为新段落的一部分%
进行排版。%
这种情况下最佳的修复方案应该是首先键入`|2|'。%
这就告诉 \TeX\ 把接下来要读入的两个字符扔掉;
并且在 \TeX\ 这样做完后,它就再次停下来让你查看新的状态。%
你将看到的如下:
\begintt
<recently read> m
|indent
<*> \hsize=4im
               \input story
?
\endtt
\1不错;`|i|' 和 `|m|' 已经被读入并且扔掉了。
但是如果你现在键入 \<return>,\TeX\ 就将 `|\input story|' 并且按照
|\hsize=4pt| 来排版 |story.tex|;那就没什么特别之处了,
因为它直接得到了几打溢出的盒子,每个都是 story 的一个音节。
还有更好的方法:你可以插入你所要的命令,即现在键入
\begintt
I\hsize=4in
\endtt
就让 \TeX\ 把 |\hsize| 改为正确值,然后 |\input story|,这就走上正路了。

%\exercise Ben ^{User} typed `|8|', not `|2|', in response to the
%error message just considered; his idea was to delete `|i|', `|m|',
%`|\input|', and the five letters of `|story|'. But \TeX's response was
%\begintt
%<*> \hsize=4im \input stor
%                          y
%\endtt
%Explain what happened.
%\answer He forgot to count the space; \TeX\ deleted `|i|', `|m|',
%`\]', `|\input|', and four letters. (But all is not lost; he can
%type `|1|' or `|2|', then \<return>, and after being prompted by `|*|' he
%can enter a new line of input.)
\exercise 用户笨笨在回应刚才出现的错误信息时键入了 `|8|' 而不是 `|2|';
他要删除 `|i|'、`|m|'、`|\input|' 以及 `|story|' 的五个字符。
但是 \TeX\ 的回应为
\begintt
<*> \hsize=4im \input stor
                          y
\endtt
看看出现了什么问题。
\answer 他忘记算上空格;\TeX\ 删除了 `|i|'、`|m|'、`\]'、`|\input|'
以及四个字母。(但没关系;他可以将输入 `|1|' 或 `|2|' 以及 \<return>,
然后在提示符 `|*|' 后他可以输入新的一行。)

%\TeX\ usually tries to recover from errors either by ignoring a command
%that it doesn't understand, or by inserting something that will keep it
%happy. For example, we saw in Chapter~6 that \TeX\ ignores an
%undefined control sequence; and we just observed that \TeX\ inserts
%`|pt|' when it needs a physical unit of measure.
%Here's another example where \TeX\ puts something in:
%\begintt
%! Missing $ inserted.
%<inserted text>
%                $
%<to be read again>
%                   ^
%l.11 the fact that 32768=2^
%                           {15} wasn't interesting
%? H
%I've inserted a begin-math/end-math symbol since I think
%you left one out. Proceed, with fingers crossed.
%\endtt
%^^{help message} ^^|Missing|
%(The user has forgotten to enclose a formula in |$| signs, and \TeX\
%has tried to recover by inserting one.) \ In this case the \<inserted text>
%is explicitly shown, and it has not yet been read; by contrast, our
%previous example illustrated a case where \TeX\ had already internalized
%the `|pt|' that it had inserted. Thus the user has a chance here to
%remove the inserted~`|$|' before \TeX\ really sees it.
 \TeX\ 通常修复错误的方式是忽略掉它不认识的命令,或者插入某些东西来去掉错误。%
例如,在第六章我们看到 \TeX\ 忽略掉一个没有定义的控制系列;
还有,我们刚刚看到,当 \TeX\ 需要一个测量的物理单位时,就插入`|pt|'。%
在下面的例子中, \TeX\ 将插入某些东西:
\begintt
! Missing $ inserted.
<inserted text>
                $
<to be read again>
                   ^
l.11 the fact that 32768=2^
                           {15} wasn't interesting
? H
I've inserted a begin-math/end-math symbol since I think
you left one out. Proceed, with fingers crossed.
\endtt
(用户落掉了封装公式的符号 |$|,\TeX\ 就通过插入它来修复。)
在这种情况下,\<inserted text> 明确给出了,并且还没有读入;
相反,在前一个例子中 \TeX\ 已经内在地插入了 `|pt|'。
因此,在 \TeX\ 实际读入 `|$|' 前用户可以有机会去掉这个插入的 `|$|'。

%What should be done? The error in this example occurred before \TeX\ noticed
%anything wrong; the characters `32768=2' have already been typeset in
%horizontal mode. There's no way to go back and cancel the past, so
%the lack of proper spacing around the `=' cannot be fixed. Our goal of
%error recovery in this case is therefore not to produce perfect
%output; we want rather to proceed in some way so that \TeX\ will pass by
%the present error and detect subsequent ones. If we were simply to hit
%\<return> now, our aim would not be achieved, because \TeX\ would
%typeset the ensuing text as a math formula: `$^{15} wasn't interesting
%\ldots$'\thinspace; another error would be detected when the paragraph is
%found to end before any closing~`|$|' has appeared. On the other hand,
%there's a more elaborate way to recover, namely to type~`|6|' and then
%`|I$^{15}$|'; this deletes `|$^{15}|' and inserts a
%correct partial formula. But that's more complicated than necessary.
%The best solution in this case is to type just~`|2|' and then go on;
%\TeX\ will typeset the incorrect equation `32768=215', but the important
%thing is that you will be able to check out the rest of the document as if
%this error hadn't occurred.
应该怎样做呢?
本例的错误出现在 \TeX\ 发现错误信息之前;
字符串`32768=2'已经在水平模式下排版完毕。%
无法回去取消过去的结果,因此在`='两边落掉的间距就无法弥补了。%
因此现在我们修复错误的目的不是得到完美的输出结果;
我们更希望用某种方式继续进行,而 \TeX\ 将越过现在的错误并且去检测下一个错误。%
\1如果我们现在直接键入 \<return>, 目的就会落空,因为 \TeX\ 将把随后的文本%
按照数学公式排版:`$^{15} wasn't interesting \ldots$'\thinspace;
当在闭符号`|$|'出现前段落就结束时就又出现一个错误。%
另一方面,有一种更精巧的修复方法,即键入`|6|'并且接着键入`|I$^{15}$|';
这就删除了`|$^{15}|'并且插入一个正确的公式部分。%
但是它比必需的要更复杂。%
这种情况下最好的解决方法是只键入`|2|', 并且接下来继续则可;
 \TeX\ 将排版出一个不正确的方程`32768=215',
但是重要的是你能验证一下文档其余的部分,就好像没出现过这个错误一样。

%\dangerexercise Here's a case in which a backslash was inadvertently omitted:
%\begintt
%! Missing control sequence inserted.
%<inserted text>
%                \inaccessible
%<to be read again>
%                   m
%l.10 \def m
%           acro{replacement}
%\endtt
%% The help message issued by TeX refers to "exercise 27.2 in The TeXbook"!
%\TeX\ needs to see a control sequence after `|\def|', so it has inserted one
%that will allow the processing to continue. \ (This control sequence is
%shown as `^|\inaccessible|', but it has no relation to any control
%sequence that you can actually specify in an error-free manuscript.) \
%If you simply hit \<return> at this point, \TeX\ will define the
%inaccessible control sequence, but that won't do you much good;
%later references to |\macro| will be undefined. Explain how to
%recover from this error so that the effect will be the same as if
%line~10 of the input file had said `|\def\macro{replacement}|'.
%\answer First delete the unwanted tokens, then insert what you want:
%Type `|6|' and then `|I\macro|'. \ (Incidentally, there's a sneaky
%way to get at the |\inaccessible| control sequence by typing
%\begintt
%I\garbage{}\let\accessible=
%\endtt
%in response to an error message like this. The author ^^{Knuth} designed
%\TeX\ in such a way that you can't destroy anything by playing such
%nasty tricks.)
\dangerexercise 下面这种情况下,无意中落掉一个反斜线:
\begintt
! Missing control sequence inserted.
<inserted text>
                \inaccessible
<to be read again>
                   m
l.10 \def m
           acro{replacement}
\endtt
% The help message issued by TeX refers to "exercise 27.2 in The TeXbook"!
在 `|\def|' 后面 \TeX\ 要读入一个控制系列,因此它插入了一个命令以继续处理。%
(这个控制系列显示的是 `^|\inaccessible|',
但它与你在正确文稿中实际给出的任何控制系列都没有关系。)
如果此时你直接键入 \<return>,那么 \TeX\ 将定义一个不能使用的控制系列,
但是这没什么好处;随后要用到的 |\macro| 就没有定义了。
看看怎样修复这个错误才能使得第 10 行的结果为 `|\def\macro{replacement}|'。
\answer 首先删除不需要的记号,然后插入你所需要的:
先输入 `|6|' 然后输入 `|I\macro|'。\allowbreak (顺便说一下,有一种鬼祟做法%
可以得到 |\inaccessible| 控制系列,即在回应这种错误信息时输入
\begintt
I\garbage{}\let\accessible=
\endtt
作者^^{Knuth}在设计 \TeX\ 时已确保使用这些诡计也不能摧毁任何东西。)

%\dangerexercise When you use the `|I|' option to respond to an error message,
%the rules of Chapter~8 imply that \TeX\ removes all spaces from the
%right-hand end of the line. Explain how you can use the `|I|'~option
%to insert a ^{space}, in spite of this fact.
%\answer `|I|\]|%|' does the trick, if |%| is a ^{comment character}.
\dangerexercise 当你用选项 `|I|' 回应错误信息时,
第 8 章的规则意味着 \TeX\ 将从行的右端去掉所有空格。
看看怎样才能绕过这个规则而用选项 `|I|' 插入一个空格?
\answer `|I|\]|%|' 就可完成此任务,假定 |%| 是一个^{注释符}。

%Some of the toughest errors to deal with are those in which you make a
%mistake on line~20 (say), but \TeX\ cannot tell that anything is amiss
%until it reaches line~25 or so. For example, if you forget a `|}|' that
%completes the argument to some macro, \TeX\ won't notice any problem
%until reaching the end of the next paragraph. In such cases you probably
%have lost the whole paragraph; but \TeX\ will usually be able to get
%straightened out in time to do the subsequent paragraphs as if
%nothing had happened. A ``^{runaway argument}'' will be displayed,
%and by looking at the beginning of that text you should be able to
%figure out where the missing `|}|' belongs.
要对付的某些最难的错误是这样的,错误出现在(比如说)第 20 行,但是 \TeX\ 直到%
第 25 行左右才告诉你出了问题。%
例如,如果你落掉了`|}|', 而它表示某些宏的参量的结束,
那么直到下一个段落的结束 \TeX\ 才会发现这个问题。%
在这种情况下,你可能已经无法挽回这个段落了;
但是 \TeX\ 一般可以及时纠正来处理随后的段落,就象什么也没有发生一样。%
将会出现一个``失控参量'', 并且通过查看此文本的开头可以找到所落掉的`|}|'在哪里。

%It's wise to remember that the first error in your document may well spawn
%spurious ``errors'' later on, because anomalous commands can inflict
%serious injury on \TeX's ability to cope with the subsequent material.
%But most of the time you will find that a single run through the
%machine will locate all of the places in which your input conflicts
%with \TeX's rules.
聪明点就知道,文档中的第一个错误可能会造成后面大量虚假的``错误'',
因为有毛病的命令会使得 \TeX\ 处理随后命令的能力遭到极大损害。%
但是大多数情况下你会发现依次运行就可以找到你的输入与 \TeX\ 规则冲突的所有地方。

%When your error is due to misunderstanding rather than mistyping,
%the situation is even more serious: \TeX's error messages will probably
%not be very helpful, even if you ask \TeX\ for help.
%If you have unknowingly redefined an important control sequence---for
%example, if you have said `|\def\box{...}|'---all sorts of strange
%disasters might occur. Computers
%aren't clairvoyant, and \TeX\ can only explain what looks wrong from
%its own viewpoint; such an explanation is bound to be mysterious
%unless you can understand the machine's attitude. The solution
%to this problem is, of course, to seek human counsel and advice;
%or, as a last resort, to read the instructions in Chapters 2, 3, \dots, 26.
\1当错误来自理解偏差而不是笔误时,问题就更严重了:
 \TeX\ 的帮助信息可能没有太多的用处,即使能给出一些。%
如果你无意中重新定义了一个重要的控制系列——例如,
如果你声明`|\def\box{...}|'——那么可能出现各种奇怪的错误。%
计算机没这个能力,并且 \TeX\ 只按照它自己的方法来处理;
如果你不了解机器的逻辑,那么这样的解释就觉得难以理解。%
当然,解决之道是请教别人;
或者使用杀手锏——看看第 2, 3, $\ldots$, 26 章中的指令。

%\newcount\serialnumber
%\def\firstnumber{\serialnumber=0 }
%\def\nextnumber{\advance \serialnumber by 1
%  \number\serialnumber)\nobreak\hskip.2em }
%\dangerexercise J. H. ^{Quick} (a student) once defined the following
%set of macros:
%\begintt
%\newcount\serialnumber
%\def\firstnumber{\serialnumber=0 }
%\def\nextnumber{\advance \serialnumber by 1
%  \number\serialnumber)\nobreak\hskip.2em }
%\endtt
%Thus he could type, for example,
%\begintt
%\firstnumber
%\nextnumber xx, \nextnumber yy, and \nextnumber zz
%\endtt
%and \TeX\ would typeset
%\firstnumber
%`\nextnumber xx, \nextnumber yy, and \nextnumber zz'.
%Well, this worked fine, and he showed the macros to his buddies.
%But several months later he received a frantic phone call; one of his
%friends had just encountered a really ^{weird error} message:
%\begintt
%! Missing number, treated as zero.
%<to be read again>
%                   c
%l.107 \nextnumber minusc
%                        ule chances of error
%?
%\endtt
%Explain what happened, and advise Quick what to do.
%\answer The `^|minus|' of `|minuscule|' was treated as part of
%the |\hskip| command in |\nextnumber|. Quick should put `|\relax|'
%at the end of his macro. \ (The ^{keywords} ^|l|, ^|plus|, |minus|,
%^|width|, ^|depth|, or ^|height| might just happen to occur in text
%when \TeX\ is reading a glue specification or a rule specification;
%designers of general-purpose macros should guard against this.
%If you get a `^|Missing number|' error and you can't guess
%why \TeX\ is looking for a number, plant the instruction
%`|\tracingcommands=1|' shortly before the error point; your log file
%will show what command \TeX\ is working~on.)
\newcount\serialnumber
\def\firstnumber{\serialnumber=0 }
\def\nextnumber{\advance \serialnumber by 1
  \number\serialnumber)\nobreak\hskip.2em }
\dangerexercise J. H. ^{Quick}(一个学生)曾经定义了下列一组宏:
\begintt
\newcount\serialnumber
\def\firstnumber{\serialnumber=0 }
\def\nextnumber{\advance \serialnumber by 1
  \number\serialnumber)\nobreak\hskip.2em }
\endtt
这样它就可以这样输入——例如
\begintt
\firstnumber
\nextnumber xx, \nextnumber yy, and \nextnumber zz
\endtt
并且 \TeX\ 应该排版出下列结果:
\firstnumber
`\nextnumber xx, \nextnumber yy, and \nextnumber zz'。%
好,这不错,他把这些宏教给几个同伴。但是几个月后,他接到一个紧急电话;
他的一个朋友遇到一个非常奇怪的错误信息:
\begintt
! Missing number, treated as zero.
<to be read again>
                   c
l.107 \nextnumber minusc
                        ule chances of error
?
\endtt
看看出现了什么问题,并且给 Quick 一个建议。
\answer \TeX\ 将 `|minuscule|' 的 `^|minus|' 视为 |\nextnumber| 中的
|\hskip| 命令的一部分。Quick 应当在这个宏的末尾加上 `|\relax|'。%
(当 \TeX\ 在读取粘连表达式或标尺表达式时,^{关键词} ^|l|、^|plus|、
|minus|、^|width|、^|depth| 或 ^|height| 可能会出现在文本中;
通用宏的设计者应当防备这种情形。如果你得到一个 `^|Missing number|' 错误,
而猜不到为何 \TeX\ 在寻找一个整数,你可以在出错点前面不远处加上
`|\tracingcommands=1|';这样日志文件中将显示 \TeX\ 正在处理的命令。)

%Sooner or later---hopefully sooner---you'll get \TeX\ to process your
%whole file without stopping once to complain. But maybe the output
%still won't be right; the mere fact that \TeX\ didn't stop doesn't mean
%that you can avoid proofreading. At this stage it's usually easy to
%see how to fix typographic errors by correcting the input. Errors of
%layout can be overcome by using methods we have discussed before:
%Overfull boxes can be cured as described in Chapter~6; bad breaks can
%be avoided by using ties or |\hbox| commands as discussed in Chapter~14;
%math formulas can be improved by applying the principles of
%Chapters 16--19.
早晚——希望是早——你用 \TeX\ 处理整个文件会顺利完成。%
但是可能输出结果还不对; \TeX\ 不停顿并不意味着你可以不用校对。%
此时,更容易通过改正输入来去掉笔误。%
版面错误可以通过前面讨论的方法来对付:
溢出的盒子可以用第六章的方法来解决;
断点不好可以通过带子命令或 |\hbox| 命令来避免,就象第十四章讨论的那样;
数学公式可以用第十六至十九章的原则来改进。

%But your output may contain seemingly inexplicable errors. For example,
%if you have specified a font at some magnification that is not supported
%by your printing software, \TeX\ will not know that there is any
%problem, but the program that converts your |dvi| file to hardcopy
%might not tell you that it has substituted an ``approximate'' font
%for the real one; the resultant spacing may look quite horrible.
但是输出结果可能包含一些看起来比较费解的错误。%
例如,如果你使用了打印设备不支持的某些放大率的字体,
那么 \TeX\ 不知道有什么问题,但是把 |dvi| 文件转换为输出结果的程序可能%
没有告诉你它用一种``近似''字体代替了你实际需要的字体;
这样就出现了非常难看的多余间距。

%If you can't find out what went wrong, try the old trick of simplifying
%your program: Remove all the things that do work, until you obtain
%the shortest possible input file that fails in the same way as the
%original. The shorter the file, the easier it will be for you or somebody
%else to pinpoint the problem.
\1如果无法找到错误的地方,就试着简化你的程序:
去掉所有正常的部分,直到得到出现同样问题的最短输入文件。%
文件越短,越容易找到错误。

%Perhaps you'll wonder why \TeX\ didn't put a blank space in some
%position where you think you typed a space. Remember that \TeX\ ignores
%^{spaces} that follow control words, when it reads your file.
%\ (\TeX\ also ignores a space after a \<number> or a \<unit of
%measure> that appears as an argument to a primitive command; but if
%you are using properly designed macros, such rules will not concern
%you, because you will probably not be using primitive commands~directly.)
可能你感到纳闷,为什么 \TeX\ 在你要输入空格的某些地方没有放置空格。%
记住,在读入文件时, \TeX\ 要忽略掉控制单词后面的空格。%
~(\TeX\ 还忽略掉作为原始命令的参量而出现的 \<number> 或 \<unit of measure>~%
后面的空格;
但是如果你使用的是设计正确的宏,那么这些规则与你无关,
因为你可能不会直接用到原始命令。)

%\ddanger On the other hand, if you are designing macros, the task of
%troubleshooting can be a lot more complicated. For example, you may
%discover that \TeX\ has emitted three blank spaces when it processed some
%long sequence of complicated code, consisting of several dozen commands.
%How can you find out where those spaces crept in? The answer is to set
%`^|\tracingcommands||=1|', as mentioned in Chapter~13. This tells \TeX\ to
%put an entry in your log file whenever it begins to execute a primitive
%command; you'll be able to see when the command is `|blank| |space|'.
\ddanger 另一方面,如果你在设计宏,那么处理问题的任务就更复杂了。%
例如,你可能发现,当 \TeX\ 处理由几打命令组成的某些复杂的长系列代码时,
可能出现了三个空格。%
怎样你才能找到这些空格的来源呢?
这就要用第十三章中提到的`^|\tracingcommands||=1|'。%
这使得 \TeX\ 只要执行一个原始命令,就在 log 文件中放一条记录;
这样你就能看到什么时候命令是`|blank| |space|'了。

%\danger Most implementations of \TeX\ allow you to ^{interrupt} the program
%in some way. This makes it possible to diagnose the causes of ^{infinite
%loops}. \TeX\ switches to ^|\errorstopmode| when interrupted; hence
%you have a chance to insert commands into the input: You can abort the
%run, or you can ^|\show| or change the current contents of control sequences,
%registers, etc. You can also get a feeling for where \TeX\ is spending most
%of its time, if you happen to be using an inefficient macro, since random
%interrupts will tend to occur in whatever place \TeX\ visits most often.
\danger 大多数 \TeX\ 的执行都允许用某些方式来中断程序。%
这就使它可以诊断出死循环的根源。%
当中断时, \TeX\ 转到 |\errorstopmode|;
因此,你就有机会把命令插入到输入中:
可以结束运行,也可以 |\show| 或改变控制系列,寄存器等的当前定义。%
如果你凑巧使用了一个效率低下的宏,你还可以觉察出 \TeX\ 把大多数时间都花%
在了什么地方了,因为随机出现的中断倾向于出现在 \TeX\ 最频繁读入的地方。

%\danger Sometimes an error is so bad that \TeX\ is forced to quit
%prematurely. For example, if you are running in ^|\batchmode| or
%^|\nonstopmode|, \TeX\ makes an ``^{emergency stop}'' if it needs
%input from the terminal; this happens when a necessary file cannot
%be opened, or when no ^|\end| command was found in the input
%document. Here are some of the messages you might get just before
%\TeX\ gives up the ghost: \enddanger
\danger 有时候错误太糟糕使得 \TeX\ 被迫过早退出。%
例如,如果在 |\batchmode| 或 |\nonstopmode| 运行,那么如果 \TeX\ 需要从终端%
得到输入时就``紧急中止''了;
当必须打开的文件被打开时,或者在输入文档中找不到命令 |\end| 时,
就出现这种情况。%
下面就是 \TeX\ 在中止运行时你可能得到的信息:\enddanger

%{\ninepoint
%\def\fatal#1. {\medbreak{\tt#1.}\par\nobreak\smallskip\noindent\ignorespaces}
%\fatal
%Fatal format file error; I'm stymied.
%^^|Fatal format file error|
%This means that the preloaded format you have specified cannot be used,
%because it was prepared for a different version of \TeX.
%\fatal
%That makes 100 errors; please try again.
%\TeX\ has scrolled past 100 errors since the last paragraph ended, so
%it's probably in an~endless ^{loop}. ^^{infinite loop}
%\fatal
%Interwoven alignment preambles are not allowed.
%^^|Interwoven alignment preambles|
%If you have been so devious as to get this message, you will understand
%it, and you will deserve no sympathy.
%\fatal
%I can't go on meeting you like this.
%^^|I can't go on|
%A previous error has gotten \TeX\ out of whack. Fix it and try again.
%\fatal
%This can't happen.
%^^|This can't happen|
%Something is wrong with the \TeX\ you are using. Complain fiercely.
%\goodbreak}
{\ninepoint
\def\fatal#1. {\medbreak{\tt#1.}\par\nobreak\smallskip\noindent\ignorespaces}
\fatal
Fatal format file error; I'm stymied.
这表示你要预载入的格式不能使用,因为它是为 \TeX\ 的另一个版本所制作的。
\fatal
That makes 100 errors; please try again.
自上一段结束起, \TeX\ 已经发现了 100 个错误,
因此它可能是个死循环。
\fatal
Interwoven alignment preambles are not allowed.
如果你错到这种情况,就应该去学习了,不值得同情。
\fatal
I can't go on meeting you like this.
前面的错误使得 \TeX\ 无法正常运行。改正它并且再试运行一次。
\fatal
This can't happen.
正在使用的对 \TeX\ 而言是错误的。问题太大。
\goodbreak}

%\danger There's also a dreadful message that \TeX\ issues only with
%great reluctance. But it can happen:
%\begintt
%TeX capacity exceeded, sorry.
%\endtt
%^^|TeX capacity exceeded|
%This, alas, means that you have tried to stretch \TeX\ too far. The
%message will tell you what part of \TeX's memory has become overloaded;
%one of the following fourteen things will be mentioned:
%\begindisplay
%|number of strings|\qquad(names of control sequences and files)\cr
%|pool size|\qquad(the characters in such names)\cr
%|main memory size|\qquad(boxes, glue, breakpoints, token lists,
%        characters, etc.)\cr
%|hash size|\qquad(control sequence names)\cr
%|font memory|\qquad(font metric data)\cr
%|exception dictionary|\qquad(hyphenation exceptions)\cr
%|input stack size|\qquad(simultaneous input sources)\cr
%|semantic nest size|\qquad(unfinished lists being constructed)\cr
%|parameter stack size|\qquad(macro parameters)\cr
%|buffer size|\qquad(characters in lines being read from files)\cr
%|save size|\qquad(values to restore at group ends)\cr
%|text input levels|\qquad(|\input| files and error insertions)\cr
%|grouping levels|\qquad(unfinished groups)\cr
%|pattern memory|\qquad(hyphenation pattern data)\cr
%\enddisplay
%The current amount of memory available will also be shown.
\danger \1还有一个可怕的信息 \TeX\ 极不情愿给出。但是也会出现:
\begintt
TeX capacity exceeded, sorry.
\endtt
唉,这意味着你所展开的 \TeX\ 太长。%
这个信息将告诉你 \TeX\ 的哪部分内存溢出了;
下面十四种情况之一会出现:
\begindisplay
|number of strings|\qquad(文件或控制系列的名称)\cr
|pool size|\qquad(在这样的名称中的字符)\cr
|main memory size|\qquad(盒子,粘连,断点,记号列,字符等。)\cr
|hash size|\qquad(控制系列的命名)\cr
|font memory|\qquad(字体度量数据)\cr
|exception dictionary|\qquad(额外的连字)\cr
|input stack size|\qquad(同时输入的资源)\cr
|semantic nest size|\qquad(要构造的未完成的列)\cr
|parameter stack size|\qquad(宏参数)\cr
|buffer size|\qquad(在要读入文件中行中的字符)\cr
|save size|\qquad(在组结束时要复原的值)\cr
|text input levels|\qquad(|\input| 文件和错误的插入对象)\cr
|grouping levels|\qquad(未完成的组)\cr
|pattern memory|\qquad(连字式样的数据)\cr
\enddisplay
当前可用的内存量也显示出来。

%\danger If you have a job that doesn't overflow \TeX's capacity, yet
%you want to see just how closely you have approached the limits,
%just set ^|\tracingstats| to a positive value before the end of your
%job. The log file will then conclude with a report on your actual
%usage of the first eleven things named above (i.e., the number of strings,
%\dots, the save size), in that order. ^^{stack positions}
%Furthermore, if you set |\tracingstats| equal to 2~or~more, \TeX\
%will show its current memory usage whenever it
%does a ^|\shipout| command. Such statistics are broken into two
%parts; `|490&5950|' means, for example, that 490 words are being used
%for ``large'' things like boxes, glue, and
%breakpoints, while 5950 words are being used for ``small'' things like
%tokens and characters.
\danger 如果有一个未溢出 \TeX\ 容量的任务,但是你只想看看离限制有多远,
那么就在任务结束前把 |\tracingstats| 设置为一个正值。%
这样在 log 文件中就对上述给出的前十一项(即,number of strings,
$\ldots$, save size)按相应顺序给出实际使用的一个报告。%
还有,如果把 |\tracingstats| 设置为大于等于 2, 那么 \TeX\ 只要执行%
一次命令 |\shipout| 就显示一次当前内存。%
这样的统计包括分为两部分;例如,`|490&5950|'表示对象盒子,粘连和断点这样%
的``大''对象用到 490 个单词,
而对象记号和字符这样的``小''对象用到 5950 个单词。

%\danger What can be done if \TeX's capacity is exceeded? All of the
%above-listed components of the capacity can be increased, provided
%that your computer is large enough; in fact, the space necessary to
%increase one component can usually be obtained by decreasing some
%other component, without increasing the total size of \TeX\null.
%If you have an especially important application, you may be able
%to convince your local system people to provide you with a special
%\TeX\ whose capacities have been hand-tailored to your needs.
%But before taking such a drastic step, be sure that you are using
%\TeX\ properly. If you have specified a gigantic paragraph or
%a gigantic alignment that spans more than one page, you should
%change your approach, because \TeX\ has to read all the way to the
%end before it can complete the line-breaking or the alignment
%calculations; this consumes huge amounts of memory space. If you have
%built up an enormous macro library, you should remember that \TeX\
%has to remember all of the replacement texts that you define; therefore
%if memory space is in short supply, you should load only the macros
%that you need. \ (See Appendices B and~D\null, for ideas on how to make
%macros more compact.)
\danger 如果超出了 \TeX\ 的容量怎么办?
如果你的计算机容量足够大,那么上面列出的所有容量分量都可用增加;
实际上,在不增加 \TeX\ 的总容量情况下,要给一个分量增加所必需的空间一般%
要减小其它某些分量。%
如果你特别重要的用处,可以请求你的本地系统给你一个可手工设置容量的特殊 \TeX。%
但是在进行这样的步骤前,首先你必须能正确使用 \TeX。%
如果你要排版超过一页的一个巨大段落或者对齐格式,那么可能要改变你的方法,
因为在 \TeX\ 完成断行或对齐计算前要把它们都读入;
这要消耗大量的内存容量。%
如果你已经建立了一个大的宏库,那么就应该记住, \TeX\ 必须记住你所定义的%
所有的替换文本;
因此如果内存空间较小,就只载入所用的宏。%
(至于怎样使宏更紧凑,见附录 B 和 D。)

%\danger Some erroneous \TeX\ programs will overflow any finite
%memory capacity. For example, after `|\def\recurse{(\recurse)}|', the
%^^{recursion} use of\/ |\recurse| will immediately bomb out:
%\begintt
%! TeX capacity exceeded, sorry [input stack size=80].
%\recurse ->(\recurse
%                     )
%\recurse ->(\recurse
%                     )
%...
%\endtt
%The same sort of error will obviously occur no matter how much you increase
%\TeX's input stack size.
\danger \1有些错误的 \TeX\ 程序将导致,无论内存容量多大都会溢出。例如,
在这个定义 `|\def\recurse{(\recurse)}|' 之后,使用 |\recurse| 就立即溢出:
\begintt
! TeX capacity exceeded, sorry [input stack size=80].
\recurse ->(\recurse
                     )
\recurse ->(\recurse
                     )
...
\endtt
不管你怎样增加 \TeX\ 的输入堆栈的容量,同样的错误总会出现。

%\ddanger The special case of ``^|save size|'' capacity exceeded is one
%of the most troublesome errors to correct, especially if you run into
%the error only on long jobs. \TeX\ generally uses up two words of save
%size whenever it performs a non-global assignment to some quantity
%whose previous value was not assigned at the same level of ^{grouping}.
%When macros are written properly, there will rarely be a need for
%more than 100 or~so things on the ``^{save stack}''; but it's possible to
%make save stack usage grow without limit if you make both local and
%^{global assignments} to the same variable. You can figure out what \TeX\
%puts on the save stack by setting ^|\tracingrestores||=1|; then your
%log file will record information about whatever is removed from the
%stack at the end of a group. For example, let |\a| stand for the
%command `|\advance\day|~|by|~|1|'; let |\g| stand for
%`|\global\advance\day|~|by|~|1|'; and consider the following commands:
%\begintt
%\day=1 {\a\g\a\g\a}
%\endtt
%The first |\a| sets |\day=2| and remembers the old value |\day=1| by putting
%it on the save stack. The first |\g| sets |\day=3|, globally; nothing needs
%to go on the save stack at the time of a global assignment. The next |\a|
%sets |\day=4| and remembers the old value |\day=3| on the save stack.
%Then |\g| sets |\day=5|; then |\a| sets |\day=6| and remembers |\day=5|.
%^^{right brace}
%Finally the `|}|' causes \TeX\ to go back through the save stack; if
%|\tracingrestores=1| at this point, the log file will get the following
%data:
%\begintt
%{restoring \day=5}
%{retaining \day=5}
%{retaining \day=5}
%\endtt
%Explanation: The |\day| parameter is first restored to its global value~5.
%Since this value is global, it will be retained, so the other saved values
%(|\day=3| and |\day=1|) are essentially ignored. Moral: If you find \TeX\
%retaining a lot of values, you have a set of macros that could cause
%the save stack to overflow in large enough jobs. To prevent this, it's
%usually wise to be consistent in your assignments to each variable that
%you use; the assignments should either be global always or local always.
\ddanger 这种``^|save size|''容量溢出的特殊情况是一个最难对付的错误,
特别是你只在大任务中出现这种错误。只要 \TeX\ 所执行的某个量的非全局赋值,
而这个量的上一个值不是在同一层次的编组中指定的,那么就要用两个字来保存大小。
当宏编写正确时,在``保存堆栈''上极少情况下才能用到 100 个以上的位置;
但如果同一个变量既有局部赋值又有全局赋值,可能导致保存所用的堆栈无限制地增长。
通过设置 |\tracingrestores||=1| 可以查看 \TeX\ 在保存堆栈上放了什么;
这样日志文件就记录下了在组结束时从堆栈中所移除内容的信息。
例如,设 |\a| 表示命令 `|\advance\day|~|by|~|1|';
|\g| 表示 `|\global\advance\day|~|by|~|1|';看看下列命令:
\begintt
\day=1 {\a\g\a\g\a}
\endtt
第一个 |\a| 设置 |\day=2| 并且把原来的值 |\day=1| 放在保存堆栈中从而保留它。
第一个 |\g| 全局地设置 |\day=3|;
在全局赋值时,堆栈上不需要什么变动。
下一个 |\a| 设置 |\day=4| 并且把原来的值 |\day=3| 放在保存堆栈中以保留。
接着,|\g| 设置 |\day=5|;接着 |\a| 设置 |\day=6| 并且保留住 |\day=5|。
最后,`|}|' 使得 \TeX\ 回退这整个保存堆栈;
如果在此处声明 |\tracingrestores=1|,那么日志文件中就得到下列数据:
\begintt
{restoring \day=5}
{retaining \day=5}
{retaining \day=5}
\endtt
注解:参数 |\day| 首先恢复其全局值 5。因为此值是全局的,所以要保留下来,
因此其它两个被保存的值(|\day=3| 和 |\day=1|)就被忽略了。
教训:如果你发现 \TeX\ 保留了很多值,那么就有那么一组宏,
在大文件中会导致保存堆栈溢出。
为了防止这种情况,最好保持对每个变量的赋值要前后一致;
赋值应该总是全局的,或者总是局部的。

%\ddanger \TeX\ provides several other kinds of tracing in addition to
%|\tracingstats| and |\tracingrestores|: We have already discussed
%|\tracingcommands| in Chapters 13 and~20, |\tracingparagraphs| in
%Chapter~14, |\tracingpages| in Chapter~15, and |\tracingmacros| in
%Chapter~20. There is also ^|\tracinglostchars|, which (if positive)
%causes \TeX\ to record each time a character has been dropped because
%it does not appear in the current font; and ^|\tracingoutput|, which
%(if positive) causes \TeX\ to display in symbolic form the contents
%of every box that is being shipped out to the ^|dvi|~file. ^^|\shipout|
%The latter allows you to see if things have been typeset properly, if
%you're trying to decide whether some anomaly was caused by \TeX\ or
%by some other software that acts on \TeX's output.
\ddanger 除了 |\tracingstats| 和 |\tracingrestores| 以外,\TeX\
还有几种跟踪命令:我们已经讨论过的有第 13 和 20 章中的 |\tracingcommands|,
第 14 章中的 |\tracingparagraphs|,第 15 章中的 |\tracingpages|,
及第 20 章中的 |\tracingmacros|。还有 |\tracinglostchars|,在其为正值时,
如果因为当前字体中不存在某个字符而漏掉了字符,\TeX\ 就给出一个记录;
以及 |\tracingoutput|,在其为正值时,\TeX\ 就以符号的形式把要输送到 |dvi|
文件中的某个盒子的内容显示出来。\1在你无法确定某个异常出在 \TeX\ 方面
还是出在处理 \TeX\ 输出的软件方面时,用后一个命令可以看看排版是否正确。

%\danger When \TeX\ displays a box as part of diagnostic output, the amount
%of data is controlled by two parameters called ^|\showboxbreadth| and
%^|\showboxdepth|. The first of these, which plain \TeX\ sets equal to~5,
%tells the maximum number of items shown per level; the second, which plain
%\TeX\ sets to~3, tells the deepest level. For example, a small box
%whose full contents are ^^{internal box format} ^^{symbolic box format}
%\begintt
%\hbox(4.30554+1.94444)x21.0, glue set 0.5
%.\hbox(4.30554+1.94444)x5.0
%..\tenrm g
%.\glue 5.0 plus 2.0
%.\tenrm || (ligature ---)
%\endtt
%will be abbreviated as follows when |\showboxbreadth=1| and |\showboxdepth=1|:
%^^{ligature} ^^{em-dash}
%\begintt
%\hbox(4.30554+1.94444)x21.0, glue set 0.5
%.\hbox(4.30554+1.94444)x5.0 []
%.etc.
%\endtt
%And if you set |\showboxdepth=0|, you get only the top level:
%\begintt
%\hbox(4.30554+1.94444)x21.0, glue set 0.5 []
%\endtt
%(Notice how `^|[]|' and `^|etc.|'~indicate that the data has been
%truncated.)
\danger 当 \TeX\ 把一个盒子作为诊断输出的一部分而显示出来时,
可以用两个参数来控制信息量,它们是 |\showboxbreadth| 和 |\showboxdepth|。%
Plain \TeX\ 把第一个设置为 5, 显示的是每个层次的项目的最大数目;
第二个设置为 3, 显示的是深度最大的层次。%
例如,当 |\showboxbreadth=1| 和 |\showboxdepth=1| 时,所有内容如下的一个小盒子
\begintt
\hbox(4.30554+1.94444)x21.0, glue set 0.5
.\hbox(4.30554+1.94444)x5.0
..\tenrm g
.\glue 5.0 plus 2.0
.\tenrm || (ligature ---)
\endtt
将简化为
\begintt
\hbox(4.30554+1.94444)x21.0, glue set 0.5
.\hbox(4.30554+1.94444)x5.0 []
.etc.
\endtt
如果设置 |\showboxdepth=0| 就只得到最顶层的信息:
\begintt
\hbox(4.30554+1.94444)x21.0, glue set 0.5 []
\endtt
(注意,`^|[]|'和`^|etc.|表明数据被截去了。)

%\danger A nonempty hbox is considered ``^{overfull}'' if its ^{glue}
%cannot shrink to achieve the specified size, provided that ^|\hbadness| is
%less than~100 or that the excess width (after shrinking by the maximum
%amount) is more than ^|\hfuzz|. It is ``^{tight}'' if its glue shrinks and
%the ^{badness} exceeds |\hbadness|; it is ``^{loose}'' if its glue
%stretches and the badness exceeds |\hbadness| but is not greater than~100;
%it is ``^{underfull}'' if its glue stretches and the badness is greater
%than |\hbadness| and greater than~100. Similar remarks apply to nonempty
%vboxes. \TeX\ prints a warning message and displays the offending box,
%whenever such anomalies are discovered. Empty boxes are never considered
%to be anomalous.
\danger 如果一个非空 hbox 的粘连不能再收缩到所要求的尺寸,
并且其 |\hbadness| 小于 100 或者(在最大收缩后)其超出的宽度%
大于 |\hfuzz|, 它就被看作是``溢出''的。%
如果粘连是收缩并且其丑度超过了 |\hbadness|, 它就是``太紧''的;
如果粘连是伸长并且其丑度超过了 |\hbadness| 而不超过 100, 就是``松散''的;
如果粘连是伸长并且其丑度超过了 |\hbadness| 且超过 100, 就是``空荡''的。%
类似讨论对非空 vbox 也适用。%
只要出现这样的问题, \TeX\ 就给出警告信息并且显示出有毛病的盒子。
空盒子从不认为有这种问题。

%\ddanger When an ^{alignment} is ``overfull'' or ``tight'' or ``loose'' or
%``underfull,'' you don't get a warning message for every aligned line;
%you get only one message, and \TeX\ displays a {\sl^{prototype row}\/}
%(or, with ^|\valign|, a {\sl prototype column\/}).
%For example, suppose you say `|\tabskip=0pt plus10pt
%\halign to200pt{&#\hfil\cr...\cr}|', ^^|\halign| and suppose that the aligned
%material turns out to make two columns of widths $50\pt$ and $60\pt$,
%respectively. Then you get the following message:
%\begintt
%Underfull \hbox (badness 2698) in alignment at lines 11--18
% [] []
%\hbox(0.0+0.0)x200.0, glue set 3.0
%.\glue(\tabskip) 0.0 plus 10.0
%.\unsetbox(0.0+0.0)x50.0
%.\glue(\tabskip) 0.0 plus 10.0
%.\unsetbox(0.0+0.0)x60.0
%.\glue(\tabskip) 0.0 plus 10.0
%\endtt
%The ``unset boxes'' in a prototype row show the individual column widths.
%In this case the ^{tabskip glue} has to stretch 3.0 times its stretchability,
%in order to reach the $200\pt$ goal, so the box is underfull. \ (According
%to the formula in Chapter~14, the badness of this situation is~2700; \TeX\
%actually uses a similar but more efficient formula, so it computes a
%badness of~2698.) \ Every line of the alignment will be underfull, but
%only the prototype row will be displayed in a warning message.
%``^{Overfull rules}'' are never appended to the lines of overfull alignments.
\ddanger 当对齐阵列是``过满的''、``紧密的''、``松散的''或``空荡的''时,
不会为每个要对齐的行给出警告信息,只能得到一个,并且 \TeX\
显示出一个{\KT{9}模板行}(在 |\valign| 中为一个{\KT{9}模板栏})。
例如,假定输入的是 `|\tabskip=0pt plus10pt \halign to200pt{&#\hfil\cr...\cr}|',
并且假定要对齐的内容把两个栏宽度分别变成 $50\pt$ 和 $60\pt$。那么就得到下列信息:
\begintt
Underfull \hbox (badness 2698) in alignment at lines 11--18
 [] []
\hbox(0.0+0.0)x200.0, glue set 3.0
.\glue(\tabskip) 0.0 plus 10.0
.\unsetbox(0.0+0.0)x50.0
.\glue(\tabskip) 0.0 plus 10.0
.\unsetbox(0.0+0.0)x60.0
.\glue(\tabskip) 0.0 plus 10.0
\endtt
\1在模板行中的``unset box''显示的是各个栏的宽度。%
在本例中,为了伸长到所要求的 $200\pt$, 制表粘连的伸长量为其伸长值的三倍,
因此此盒子是空荡的。%
(按照第十四章的公式,此情形下的丑度为 2700;
 \TeX\ 实际用的是一个类似且更高效的公式,因此计算出的丑度为 2698。)
对齐的每个行都是空荡的,但是只有模板行在警告信息中给出了。%
对溢出对齐的行不再添加``溢出的线''。

%\ddanger The |\tracing...|\ commands put all of their output into your log
%file, unless the ^|\tracingonline| parameter is positive; in the latter
%case, all diagnostic information goes to the terminal as well as to the
%log file.  Plain \TeX\ has a ^|\tracingall| macro that turns on the
%maximum amount of tracing of all kinds. It not only sets~up
%|\tracingcommands|, |\tracingrestores|, |\tracingparagraphs|, and so on,
%it also sets |\tracingonline=1|, and it sets ^|\showboxbreadth| and
%^|\showboxdepth| to extremely high values, so that the entire contents of
%all boxes will be displayed.
\ddanger 如果参数 |\tracingonline| 不是正值,那么 |\tracing...|%
命令的所有输出都放在日志文件中;如果是正值,
所有诊断信息都同时放在终端和日志文件中。
Plain \TeX\ 有一个宏 |\tracingall|,它把各种诊断信息全部给出。
它不但设置了 |\tracingcommands|、|\tracingrestores|、|\tracingparagraphs| 等,
还设置 |\tracingonline=1|,并且设置 |\showboxbreadth| 和 |\showboxdepth|
很大的值,这样就能把所有盒子的全部内容都显示出来。

%\ddanger Some production versions of \TeX\ have been streamlined for
%speed. These implementations don't look at the values of the parameters
%|\tracingparagraphs|, |\tracingpages|, |\tracingstats|, and
%|\tracingrestores|, because \TeX\ runs faster when it doesn't have
%to maintain statistics or keep tabs on whether tracing is required.
%If you want all of \TeX's diagnostic tools, you should be sure to
%use the right version.
\ddanger 某些 \TeX\ 版本在速度上有所改进。%
这些运行看不到参数 |\tracingparagraphs|, |\tracingpages|, |\tracingstats| 和 %
|\tracingrestores| 的值,因为当 \TeX\ 不必进行统计或者不理会诊断时运行得更快。%
如果要使用所有的诊断工具,就要用选对版本。

%\ddanger If you set ^|\pausing||=1|, \TeX\ will give you a chance to edit
%each line of input as it is read from the file. In this way you can
%make temporary patches (e.g., you can insert |\show...|\ commands)
%while you're troubleshooting, without changing the actual contents
%of the file, and you can keep \TeX\ running at human speed.
\ddanger 如果设置 |\pausing||=1|, 那么 \TeX\ 将在读入文件时允许你编辑每一行。%
用这种方法,当你要处理问题时,就可以临时补救一下(比如插入命令 |\show...|)%
而不改变文件的实际内容,并且可以让 \TeX\ 按照人为的速度来运行。

%Final hint: When working on a long manuscript, it's best to prepare
%only a few pages at a time. Set up a ``^{galley}'' file and a ``^{book}''
%file, and enter your text on the galley file. \ (Put control
%information that sets up your basic format at the beginning of this
%file; an example of |galley.tex| appears in Appendix~E\null.) \
%After the galleys come out looking right, you can append them to the
%book file; then you can run the book file through \TeX\ occasionally,
%in order to see how the pages really fit together. For example,
%when the author prepared this manual, he did one chapter at a time,
%and the longer chapters were split into subchapters.
最后的提示:
当要制作一个大文稿时,最好每次只处理几页。%
设置一个临时文件和一个总文件,并且把文本放在分文件中。%
(把控制基本格式的控制信息放在此文件的开头;在附录 E 中有一个例子 |galley.tex|。)
在临时文件正确后,再把它添加到总文件中;
这样你只需要偶尔编译总文件看看页面的合并情况。%
例如,当作者写此手册时,每次处理一章,
并且把长的章分为几个片段。

%\ddangerexercise Final exercise: Find all of the ^{lies} in this
%manual, and all of the ^{jokes}.
%\answer If this exercise isn't just a joke, the title of this
%appendix is a lie.
\ddangerexercise 最后的练习:找出本手册中所有的谎话和玩笑。
\answer 如果这个练习不是一个玩笑,那么这个附录的标题就是一句谎话。

%\line{Final exhortation: G{\sc O} {\sc FORTH} now and create
%{\sl masterpieces of the publishing art!\/}}
\line{\HT{10}最后的倡议:出发,去创造出版业之杰作吧!}

\endchapter

Who can understand his errors?
\author ^^{Biblical}{\sl Psalm 19\thinspace:\thinspace12\/} (c.~1000 B.C.)

\bigskip

It is one thing, to shew a Man that he is in an Error,
and another, to put him in possession of Truth.
\author JOHN ^{LOCKE}, {\sl An Essay Concerning Humane Understanding\/} (1690)
  % bk 4 ch 7 sec 11

\vfill\eject\byebye
