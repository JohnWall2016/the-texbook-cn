% -*- coding: utf-8 -*-

\input macros

%\beginchapter Chapter 7. How \TeX\ Reads\\What You Type
\beginchapter Chapter 7. 工作原理

\origpageno=37

%We observed in the previous chapter that an input manuscript is expressed
%in terms of ``lines,'' but that these lines of input are essentially
%independent of the lines of output that will appear on the finished pages.
%Thus you can stop typing a line of input at any place that's convenient for
%you, as you prepare or edit a file. A few other related rules have also
%been mentioned:
\1在上一章我们看到,输入的文稿用``行''的方式来表达,
但是这些输入的行和最后出现在页面中的行本质上是不同的。%
因此,当输入或编辑文件时,你可以随心所欲地换行。%
这里要提出几个相关的规则:

%\medskip
%\item\bull A $\langle\hbox{return}\rangle$ is like a space.
\medskip
\item\bull 一个 $\langle\hbox{return}\rangle$与一个空格类似。

%\smallskip
%\item\bull Two spaces in a row count as one space.
\smallskip
\item\bull 一行中的两个空格看作一个。

%\smallskip
%\item\bull A blank line denotes the end of a paragraph.
\smallskip
\item\bull 一个空行表示一段的结束。

%\medskip
%\noindent Strictly speaking, these rules are contradictory: A blank line
%is obtained by typing $\langle\hbox{return}\rangle$ twice in a row,
%and this is different from typing two spaces in a row. Some day you might want
%to know the {\sl real\/} rules. In this chapter and the next, we shall study
%the very first stage in the transition from input to output.
\medskip
\noindent 严格说,这些规则是矛盾的:
在一行内键入两次 $\langle\hbox{return}\rangle$ 就得到一个空行,
而且这与在一行键入两个空格不同。%
有一天你可能想了解{\KT{10}真正的}规则。%
在本章和下一章,我们将讨论从输入到输出的相当初始的阶段。

%\smallskip
%In the first place, it's wise to have a precise idea of what your keyboard
%sends to the machine. There are 256 characters that \TeX\ might encounter at
%each step, in a file or in a line of text typed directly on your terminal. These
%256~characters are classified into 16 categories numbered 0 to 15:
%\begindisplay \def\\{\hfill}
%\hfil\hidewidth\it Category\hidewidth&\it \qquad Meaning\hidewidth\cr
%\noalign{\smallskip}
%\\0&Escape character&(|\| in this manual)\cr
%\\1&Beginning of group&(|{| in this manual)\cr
%\\2&End of group&(|}| in this manual)\cr
%\\3&Math shift&(|$| in this manual)\cr
%\\4&Alignment tab&(|&| in this manual)\cr
%\\5&End of line&(\<return> in this manual)\cr
%\\6&Parameter&(|#| in this manual)\cr
%\\7&Superscript&(|^| in this manual)\cr
%\\8&Subscript&(|_| in this manual)\cr
%\\9&Ignored character&(\<null> in this manual)\cr
%10&Space&(\] in this manual)\cr
%11&Letter&(|A|, \dots, |Z| and |a|, \dots, |z|)\cr
%12&Other character&(none of the above or below)\cr
%13&Active character&(|~| in this manual)\cr
%14&Comment character&(|%| in this manual)\cr
%15&Invalid character&(\<delete> in this manual)\cr
%\enddisplay
%^^{escape character}
%^^{begin-group character}
%^^{end-group character}
%^^{math mode character}
%^^{alignment tab}
%^^{parameter}
%^^{superscript}
%^^{subscript}
%^^{ignored character}
%^^{space}
%^^{letter}
%^^{other character}
%^^{active character}
%^^{comment character}
%^^{invalid character}
%^^{category codes, table}
%It's not necessary for you to learn these code numbers; the point is only that
%\TeX\ responds to 16~different types of characters. At first this manual led
%you to believe that there were just two types---the escape character and the
%others---and then you were told about two more types, the grouping
%symbols |{| and~|}|. In Chapter~6 you learned two more: |~| and~|%|.
%Now you know that there are really~16. This is the whole truth of the
%matter; no more types remain to be revealed.  The category code for any
%character can be changed at any time, but it is usually wise to stick to a
%^^{reserved character} ^^{special character table} ^^\<null> ^^\<delete>
%particular scheme.
\smallskip
首先,明智之举是对键盘发送到计算机中的准确内容有一个了解。%
\TeX\ 可能见到的直接从键盘上键入文件或行中的字符有 256 个。%
这 256 个字符被分为 16 类,数字是从 0 到 15:
\begindisplay \def\\{\hfill}
\hfil\hidewidth\it (类)Category\hidewidth&\it \qquad 意义\hidewidth\cr
\noalign{\smallskip}
\\0&转义符 &(在本手册为 |\|)\cr
\\1&组开始&(在本手册为 |{|)\cr
\\2&组结束&(在本手册为 |}|)\cr
\\3&数学环境&(在本手册为 |$|)\cr
\\4&表格对齐&(在本手册为 |&|)\cr
\\5&换行&(在本手册为 \<return>)\cr
\\6&参数&(在本手册为 |#|)\cr
\\7&上标&(在本手册为 |^|)\cr
\\8&下标&(在本手册为 |_|)\cr
\\9&可忽略的字符&(在本手册为 \<null>)\cr
10&空格&(在本手册为 \])\cr
11&字母&(|A|, \dots, |Z| 和 |a|, \dots, |z|)\cr
12&其它字符&(不在上下文的其它字符)\cr
13&活动符&(在本手册为 |~|)\cr
14&注释符&(在本手册为 |%|)\cr
15&无用符&(在本手册为 \<delete>)\cr
\enddisplay
对你而言,不必掌握这些代码数;
要点只是 \TeX\ 对 16 类不同字符所做的处理。%
首先,本手册中出现的只有两种类型——转义符和其它字符——并且接着你%
再学会两种类型,编组符号 |{| 和 |}|。%
在第六章,又出现两种:|~| 和 |%|。%
现在你清楚了,其实有 16 类。%
这才是问题的全部;
不会再出现更多类型了。%
任意字符的类代码可以在任何时候改变,
但是保持一个特殊的方案通常比较明智。

%The main thing to bear in mind is that each \TeX\ format reserves certain
%characters for its own special purposes. For example, when you are using plain
%\TeX\ format (Appendix~B\null), you need to know that the ten characters
%\begintt
%\ { } $ & # ^ _ % ~
%\endtt
%cannot be used in the ordinary way when you are typing;
%^^{special characters}
%^^{backslash}^^{left brace}^^{right brace}^^{dollar sign}^^{ampersand}
%^^{hash mark}^^{hat}^^{underline}^^{percent}^^{tilde}
%^^{single-character control sequences}
%each of them will cause \TeX\ to do something special, as explained elsewhere
%in this book. If you really need these symbols as part of your manuscript,
%plain \TeX\ makes it possible for you to type
%\begindisplay
%|\$| for \$,\qquad |\%| for \%,\qquad |\&| for \&,\qquad
%|\#| for \#,\qquad |\_| for \_\thinspace;
%\enddisplay
%the |\_| symbol is useful for {\it compound\_identifiers\/} in computer
%^^{identifiers} ^^{computer programs}
%programs. In mathematics formulas you can use |\{| and |\}| for $\{$ and
%$\}$, while ^|\backslash| produces a ^{reverse slash}; for example,
%\begindisplay
%`|$\{a \backslash b\}$|'\quad yields\quad`$\{a\backslash b\}$'.
%\enddisplay
%Furthermore |\^| produces a circumflex accent (e.g., `|\^e|' yields
%`\^e'\thinspace); and |\~| yields a tilde accent (e.g., `|\~n|' yields
%`\~n'\thinspace).
\1要记住的主要问题是,每个 \TeX\ 格式都为各自的特殊目的而保留一定的字符。%
例如,当你使用 plain \TeX\ 格式(附录 B)时,就需要知道,下列 10 个字符
\begintt
\ { } $ & # ^ _ % ~
\endtt
在键入时不能按普通方法使用;
它们的每个都使 \TeX\ 做某些特殊的事情,在本书的某些地方给出解释。%
如果你实在需要输出这些字符,plain \TeX\ 提供了下列方法:
\begindisplay
|\$| 得到 \$,\qquad |\%| 得到 \%,\qquad |\&| 得到 \&,\qquad
|\#| 得到 \#,\qquad |\_| 得到 \_\thinspace;
\enddisplay
符号 |\_| 用在计算机程序的{\it compound\_identifiers\/}。%
在数学公式中,可以使用 |\{| 和 |\}| 来得到 $\{$ 和 $\}$,
而且 |\backslash| 得到一个反斜线;
例如,
\begindisplay
`|$\{a \backslash b\}$|'\quad 得到 \quad`$\{a\backslash b\}$'.
\enddisplay
还有,|\^| 得到的是一个 circumflex 重音(比如,`|\^e|'得到的是`\^e'\thinspace);
并且 |\~| 得到的是一个波浪重音(比如,`|\~n|'得到的是
`\~n'\thinspace)。

%\exercise What horrible errors appear in the following sentence?
%^^{Procter} ^^{Gamble}
%\begintt
%Procter & Gamble's stock climbed to $2, a 10% gain.
%\endtt
%\answer Three forbidden characters were used. One should type
%\begintt
%Procter \& Gamble's ... \$2, a 10\% gain.
%\endtt
%(Also the facts are wrong.)
\exercise 在下列句子中,会出现什么讨厌的问题?
\begintt
Procter & Gamble's stock climbed to $2, a 10% gain.
\endtt
\answer 用到三个特殊字符。应该这样键入
\begintt
Procter \& Gamble's ... \$2, a 10\% gain.
\endtt
(另外此事实是错误的。)

%\exercise Can you imagine why the designer of plain \TeX\ decided not
%to make `|\\|' the control sequence for reverse slashes?^^{backslash}
%\answer Reverse slashes (backslashes) are fairly uncommon in formulas or
%text, and |\\| is very easy to type; it was therefore felt best not to
%reserve |\\| for such limited use. Typists can define |\\| to be whatever
%they want (including |\backslash|).
\exercise 你能想像一下 plain \TeX\ 的设计者为什么不用`|\\|'这个控制系列代表反斜线吗?
\answer 反斜线在正文和公式中都很少见到,而 |\\| 又很容易键入;
因此最好是不用 |\\| 表示这种很少使用的字符。
排版者可以将 |\\| 定义为他们所需的任何东西(包括 |\backslash|)。

%\danger When \TeX\ reads a line of text from a file, or a line of text that
%you entered directly on your terminal, it converts that text into a list of
%``^{tokens}.'' A token is either (a)~a single character with an attached
%category code, or (b)~a control sequence. For example, if the normal
%conventions of plain \TeX\ are in force, the text `|{\hskip 36 pt}|' is
%converted into a list of eight tokens:
%\begindisplay
%|{|$_1$\quad\cstok{hskip}\quad|3|$_{12}$\quad|6|$_{12}$\quad
%  \]$_{10}$\quad|p|$_{11}$\quad|t|$_{11}$\quad|}|$_{2}$
%\enddisplay
%The subscripts here are the category codes, as listed earlier: 1 for
%``beginning of group,'' 12 for ``other character,'' and so on. The
%\cstok{hskip} doesn't get a subscript, because it represents a control
%sequence token instead of a character token. Notice that the space after
%|\hskip| does not get into the token list, because it follows a
%^{control word}.
\danger 当 \TeX\ 从文件中读入一行或直接从终端上读入一行时,它把文本转变成一列``记号''。%
一个记号可以是 (a) 一个指定类代码的单个字符,或者是 (b) 一个控制系列。%
例如,在 plain \TeX\ 的约定下,文本`|{\hskip 36 pt}|'转变为一个 8 个记号的列:
\begindisplay
|{|$_1$\quad\cstok{hskip}\quad|3|$_{12}$\quad|6|$_{12}$\quad
  \]$_{10}$\quad|p|$_{11}$\quad|t|$_{11}$\quad|}|$_{2}$
\enddisplay
这里的下标是如前所述的类代码:
1 是``组开始'', 12 是``其它字符'', 诸如此类。%
\cstok{hskip} 没有赋予下标,因为它表示一个控制系列记号而不是一个字符记号。%
注意,记号列中没有 |\hskip| 后的空格,因为它所跟的是一个控制词。

%\danger It is important to understand the idea of token lists, if you want
%to gain a thorough understanding of \TeX, and it is convenient to learn
%the concept by thinking of \TeX\ as if it were a living organism. The
%individual lines of input in your files are seen only by \TeX's ``eyes''
%and ``mouth''; but after that text has been gobbled up, it is sent to
%\TeX's ``stomach'' in the form of a token list, and the digestive processes
%that do the actual typesetting are based entirely on tokens. As far as the
%stomach is concerned, the input flows in as a stream of tokens, somewhat
%as if your \TeX\ manuscript had been typed all on one extremely long line.
\danger 如果你想对 \TeX\ 有一个完整的理解,那么理解记号列表的思想很重要的,
把 \TeX\ 的思想看作一个生物体对掌握 \TeX\ 的思想很方便。%
文件中输入的每个行只能被``眼睛''和``嘴巴''遇见;
但当文本被吞下后,被按照记号列表的方式送到``胃''里,
而且处理实际排版的消化过程是完全按照记号来进行的。%
就胃而言,输入内容就象记号流一样流出,好象你的 \TeX\ 文稿被一起键入在一个很长的行中。

%\danger You should remember two chief things about \TeX's tokens: (1)~A
%control sequence is considered to be a single object that is no longer
%composed of a sequence of symbols. Therefore long control sequence names
%are no harder for \TeX\ to deal with than short ones, after they have been
%replaced by tokens. Furthermore, spaces are not ignored after control
%sequences inside a token list; the ignore-space rule applies only in an
%input file, during the time that strings of characters are being
%tokenized.  (2)~Once a category code has been attached to a character
%token, the attachment is permanent. For example, if character `|{|' were
%suddenly declared to be of category~12 instead of category~1, the
%characters `|{|$_1$' already inside token lists of \TeX\ would still
%remain of category 1; only newly made lists would contain `|{|$_{12}$'
%tokens. In other words, individual characters receive a fixed
%interpretation as soon as they have been read from a file, based on the
%category they have at the time of reading. Control sequences are
%different, since they can change their interpretation at any time.  \TeX's
%digestive processes always know exactly what a character token signifies,
%because the category code appears in the token itself; but when the
%digestive processes encounter a control sequence token, they must look up
%the current definition of that control sequence in order to figure out
%what it means.
\danger \1还记得 \TeX\ 记号的两个主要部分吧:
(1) 一个控制系列被看作一个单个对象,它不是由一系列符号组成的。%
因此,当控制系列用记号代替后,\TeX\ 处理长名字与短的是一样的。%
还有,在记号列表中间的空格不能忽略;因为忽略空格的规则只适用于输入文件,
在那期间,字符串被变成记号。%
(2) 一个字符记号只能有一个类代码,这个指定是永久性的。%
例如,如果字符`|{|'突然被声明为类代码为 12 而不是 1, 已经在 \TeX\ 的记号列表中%
的字符`|{|$_1$'的类代码仍然是 1; 只有新的列表才包含记号`|{|$_{12}$'。%
换句话说,一旦某个字符从文件中被读出来,那么它的解释就固定了,
就是它读入时的类代码。%
控制系列却不同,因为它们可以在任何时间改变它们的解释。%
\TeX\ 的消化过程总精确地知道一个字符记号表示什么,
因为类代码出现在记号自己身上;
但是当消化过程遇到一个控制系列记号,为了确定它的意思,必须找到控制系列当前的定义。

%\ddangerexercise Some of the category codes 0 to 15 will never appear as
%subscripts in character tokens, because they disappear in \TeX's mouth.
%For example, characters of category 0 (escapes) never get to be tokens.
%Which categories can actually reach \TeX's stomach?
%\answer 1, 2, 3, 4, 6, 7, 8, 10, 11, 12, 13. ^{Active characters} (type 13)
%are somewhat special; they behave like control sequences in most cases
%(e.g., when you say `^|\let||\x=~|' or `^|\ifx||\x~|'), but they behave like
%character tokens when they appear in the token list of\/ ^|\uppercase|
%or ^|\lowercase|, and when unexpanded after ^|\if| or ^|\ifcat|.
\ddangerexercise 0 到 15 中某些类代码从不出现在字符记号的下标上,
因为它们被 \TeX\ 的嘴吃掉了。%
例如,类代码为 0 的字符(转义符)从未出现在记号中。%
哪些类实际出现在 \TeX\ 的胃中?
\answer 1, 2, 3, 4, 6, 7, 8, 10, 11, 12, 13。
^{活动符}(第 13 类)有些特殊:
在多数情形它们很像控制系列(比如你可以用 `^|\let||\x=~|' 或 `^|\ifx||\x~|'),
但是在 ^|\uppercase| 或 ^|\lowercase| 的记号列中,
以及在 ^|\if| 或 ^|\ifcat| 后不展开时,它们很像字符记号。

%\ddanger There's a program called ^|INITEX| that is used to install
%\TeX, starting from scratch; |INITEX| is like \TeX\ except that it can
%do even more things. It can compress ^{hyphenation} patterns into special
%tables that facilitate rapid hyphenation, and it can produce ^{format}
%files like `|plain.fmt|' from `|plain.tex|'.  But |INITEX| needs extra
%space to carry out such tasks, so it generally has less memory available
%for typesetting than you would expect to find in a production version of \TeX.
\ddanger 有一个叫 |INITEX| 的程序,它被用来从最开始来安装 \TeX;
|INITEX| 类似于 \TeX, 只是它的功能更\hbox{多。}%
它可以把连字符模式压缩到一个特殊的表中从而使得连字化更快,
并且它可以从`|plain.tex|'得到类似`|plain.fmt|'的格式文件。%
但是运行此类任务时,|INITEX| 需要额外的空间,因此,
在 \TeX\ 的安装后的版本中,一般排版可用的内存比你所预期的要少。

%\ddanger When |INITEX| begins, it knows nothing
%but \TeX's primitives. All 256~characters are initially of category~12,
%except that ^\<return> has category~5,
%^\<space> has category~10, ^\<null> has category~9, ^\<delete> has category~15,
%the 52 letters |A|$\,\ldots\,$|Z| and |a|$\,\ldots\,$|z| have category~11,
%|%| and~|\| have the respective categories 14 and~0. ^^{backslash}^^{percent}
%It follows that |INITEX| is initially incapable of carrying out some of
%\TeX's primitives that depend on grouping; you can't use |\def| or |\hbox|
%until there are characters of categories 1 and~2. The format in
%Appendix~B begins with ^|\catcode| commands to provide characters of the
%necessary categories; e.g.,
%\begintt
%\catcode`\{=1
%\endtt
%assigns category 1 to the |{| symbol. The |\catcode| operation is like
%many other primitives of \TeX\ that we shall study later; by modifying
%internal quantities like the category codes, you can adapt \TeX\ to a wide
%variety of applications.
\ddanger 当 |INITEX| 运行时,它只认识 \TeX\ 的原始控制系列。%
所有 256 个字符的类代码开始都是 12, 除了 \<return> 的是 5,
\<space> 的是 10, \<null> 的是 9, \<delete> 的是 15,
52 个字母 |A|$\,\ldots\,$|Z| 和 |a|$\,\ldots\,$|z| 的是 11,
|%| 和 |\| 的分别是 14 和 0。%
由此,|INITEX| 最初不能处理依赖于编组的 \TeX\ 某些原始控制系列;
在类代码为 1 和 2 的字符出现之前,不能使用 |\def| 或 |\hbox|。%
以 |\catcode| 开始的附录 B 给出了这些必需的类代码;
比如,
\begintt
\catcode`\{=1
\endtt
把符号 |{| 的类代码指定为 1。%
操作 |\catcode| 类似于我们后面要讨论的许多其它 \TeX\ 的原始控制系列;
通过修改象类代码这样的内部量,可以使 \TeX\ 的应用相当广泛。

%\ddangerexercise Suppose that the commands
%\begintt
%\catcode`\<=1 \catcode`\>=2
%\endtt
%appear near the beginning of a group that begins with `|{|'; these
%specifications instruct \TeX\ to treat |<| and |>| as group delimiters.
%According to \TeX's rules of locality, the characters |<| and |>| will
%revert to their previous categories when the ^{group} ends. But should the
%group end with |}| or~with~|>|\thinspace?
%\answer It ends with either |>| or |}| or any character of category 2;
%then the effects of all |\catcode| definitions within the group are wiped
%out, except those that were ^|\global|. \TeX\ doesn't have any built-in
%knowledge about how to pair up particular kinds of grouping characters.
%New category codes take effect as soon as a |\catcode| assignment has been
%digested. For example,
%\begintt
%{\catcode`\>=2 >
%\endtt
%is a complete group. But without the space after `|2|' it would not be
%complete, since \TeX\ would have read the~`|>|' and converted it to a
%token before knowing what category code was being specified; \TeX\ always
%reads the token following a constant before evaluating that ^{constant}.
\ddangerexercise 假定命令
\begintt
\catcode`\<=1 \catcode`\>=2
\endtt
出现在以`|{|'开始的组的开头;
这些规定告诉 \TeX\ 把 |<| 和 |>| 看作组的分隔符。%
按照 \TeX\ 的局部性规则,当组结束时,字符 |<| 和 |>| 的类代码将复原回去。%
但是组应该以 |}| 还是 |>| 来结束?
\answer 用 |>| 或者 |}| 或者任何类别码为 2 的字符都可以结束编组;
然后该编组内部的 |\catcode| 定义的效果将被取消,除非那些用 ^|\global| 定义的。
\TeX\ 本身并不在乎组开始符和组结束符的类型是否相配。
|\catcode| 赋值一经消化,新的类别码就立即生效。例如,
\begintt
{\catcode`\>=2 >
\endtt
是一个完整的编组。但是如果去掉 `|2|' 之后的空格,该编组将是不完整的,
因为 \TeX\ 在类别码指定之前就已经读入 `|>|' 并将其转化为记号;
在对^{常数}求值之前 \TeX\ 总是先读取常数之后的那个记号。

%\ddanger Although control sequences are treated as single objects,
%\TeX\ does provide a way to break them into lists of character tokens:
%If you write ^|\string||\cs|,
%where |\cs| is any control sequence, you get the list of characters for that
%control sequence's name. For example, |\string\TeX| produces four tokens:
%|\|$_{12}$, |T|$_{12}$, |e|$_{12}$, |X|$_{12}$. Each character in this token
%list automatically gets category code~12 (``other''),
%including the ^{backslash} that |\string| inserts to represent an escape
%character.  However, category~10 will be assigned to the character `\]'
%(blank ^{space}) if a space character somehow sneaks into the name of a
%control sequence.
\ddanger \1虽然控制系列被看作单个对象,但是 \TeX\ 的确提供了一种方法来把它分成一列%
字符记号:
如果输入 |\string||\cs|, 其中 |\cs| 是任意控制系列,那么就得到那个控制系列的名字的%
一个列表。%
例如,|\string\TeX| 就得到 4 个记号:
|\|$_{12}$, |T|$_{12}$, |e|$_{12}$, |X|$_{12}$。
在这个记号列表中的每个字符自动得到类代码 12~(``其它字符''),
包括表示转义符的反斜线。%
但是,如果字符空格无缘无故出现在控制系列的名字中,字符`\]'~(空\hbox{格)}的类代码将指定为 10。

%\ddanger Conversely, you can go from a list of character tokens to a
%control sequence by saying `^|\csname|\<tokens>^|\endcsname|'. The tokens
%that appear in this construction between |\csname| and |\endcsname| may
%include other control sequences, as long as those control sequences
%ultimately expand into characters instead of \TeX\ primitives; the final
%characters can be of any category, not necessarily letters. For example,
%`|\csname TeX\endcsname|' is essentially the same as `|\TeX|'; but
%`|\csname\TeX\endcsname|' is illegal, because |\TeX| expands into tokens
%containing the ^|\kern| primitive. Furthermore,
%`|\csname\string\TeX\endcsname|' will produce the unusual control sequence
%`|\\TeX|', i.e., the token \cstok{\char`\\TeX}, which you can't ordinarily
%write.
\ddanger 反过来,用命令`|\csname|\<tokens>|\endcsname|', 可以把一个字符记号列表%
变成一个控制系列。%
出现在 |\csname| 和 |\endcsname| 之间的这个指令中的记号可以包括其它控制系列,
只要这些控制系列最后展开成的是字符而不是 \TeX\ 的原始控制系列;
最后的字符可以是任意类,不必是字母。%
例如,`|\csname TeX\endcsname|'本质上和 |\TeX| 是一样的;
但是`|\csname\TeX\endcsname|'是不合法的,
因为 |\TeX| 展开为记号时包含原始控制系列 |\kern|。
还有,`|\csname\string\TeX\endcsname|'就得到一个不常见的控制系列`|\\TeX|', 即,
记号 \cstok{\char`\\TeX}, 而你不能用一般的方法来定义它。

%\ddangerexercise Experiment with \TeX\ to see what |\string| does when it
%is followed by an ^{active character} like |~|. \ (Active characters behave
%like control sequences, but they are not prefixed by an escape.) \ What
%is an easy way to conduct such experiments online? What control sequence
%could you put after |\string| to~obtain the single character
%token~|\|$_{12}$?
%\answer If you type `|\message{\string~}|' and `|\message{\string\~}|', \TeX\
%responds with `|~|' and `|\~|', respectively. ^^|\message|
%To get |\|$_{12}$ from |\string| you therefore need to make backslash an
%active character. One way to do this is
%\begintt
%{\catcode`/=0 \catcode`\\=13 /message{/string\}}
%\endtt
%(The ``^{null control sequence}'' that you get when there are no
%tokens between |\csname| and |\endcsname| is not a solution to this exercise,
%because |\string| converts it to `|\csname\endcsname|'. There is, however,
%another solution: If \TeX's ^|\escapechar| parameter---which will be
%explained in one of the next dangerous bends---is negative or greater
%than~255, then `|\string\\|' works.)
\ddangerexercise 用 \TeX\ 做个实验,看看 |string| 后面跟一个象 |~| 的活动符会%
出现什么情况。(活动符的表现象控制系列,但是前面没有转义符。)
在行中做这个实验的简单方法是什么?
为了得到单个字符记号 |\|$_{12}$, |\string|后面应该跟什么?
\answer 如果你键入 `|\message{\string~}|' 或 `|\message{\string\~}|',
\TeX\ 将分别给出 `|~|' 和 `|\~|'。^^|\message|%
因此要用 |\string| 得到 |\|$_{12}$ 你需要让反斜线变成活动符。其中一种做法是
\begintt
{\catcode`/=0 \catcode`\\=13 /message{/string\}}
\endtt
( 当 |\csname| 和 |\endcsname| 之间无任何记号时所得到的``^{空控制系列}''%
不是此练习的解答,因为 |\string| 将它转换为 `|\csname\endcsname|'。
然而,确实还有另一个解答:如果 \TeX\ 的 ^|\escapechar| 参数——%
这会在后面某个险弯处解释——小于零或者大于 255,则 `|\string\\|' 也可以。)

%\ddangerexercise What tokens does
%`|\expandafter\string\csname a\string\   b\endcsname|' produce?
%(There are three spaces before the |b|. Chapter~20 explains ^|\expandafter|.)
%\answer |\|$_{12}$ |a|$_{12}$ |\|$_{12}$ \]$_{10}$ |b|$_{12}$.
\ddangerexercise `|\expandafter\string\csname a\string\   b\endcsname|'得到什么记号?
(在 |b| 前面有 3 个空格。第二十章给出了 |\expandafter| 的解释。)
\answer |\|$_{12}$ |a|$_{12}$ |\|$_{12}$ \]$_{10}$ |b|$_{12}$。

%\ddangerexercise When |\csname| is used to define a control sequence for
%the first time, that control sequence is made equivalent to |\relax|
%until it is redefined. Use this fact to design a macro |\ifundefined#1|
%such that, for example,
%\begindisplay
%|\ifundefined{TeX}|\<true text>|\else|\<false text>|\fi|
%\enddisplay
%expands to the \<true text> if\/ |\TeX| hasn't previously been defined,
%or if\/ |\TeX| has been |\let| equal to |\relax|; it should expand
%to the \<false text> otherwise. ^^|\ifundefined|
%\answer |\def\ifundefined#1{\expandafter\ifx\csname#1\endcsname\relax}|%
%\hfil\break Note that a control sequence like this must be used with care;
%it cannot be included in ^{conditional} text, because the |\ifx| will not
%be seen when |\ifundefined| isn't expanded.
\ddangerexercise 当 |\csname| 用来第一次定义控制系列时,
那个控制系列在重新被定义前,等价于 |\relax|。%
利用这个结果来设计一个宏 |\ifundefined#1|~, 比如,使得
\begindisplay
|\ifundefined{TeX}|\<true text>|\else|\<false text>|\fi|
\enddisplay
它在 |\TeX| 没有被定义,或者被 |\let| 等于 |\relax| 时展开为 \<true text>;
其它时候展开为 \<false text>。
\answer |\def\ifundefined#1{\expandafter\ifx\csname#1\endcsname\relax}|%
\hfil\break 注意像这样的控制系列必须小心使用;
它不能包含在^{条件}文本中,因为在 |\ifundefined| 被展开之前看不到 |\ifx|。

%\ddanger In the examples so far, |\string| has converted control sequences
%into lists of tokens that begin with |\|$_{12}$. But this backslash token isn't
%really hardwired into \TeX; there's a parameter called ^|\escapechar| that
%specifies what character should be used when control sequences are output
%as text. The value of\/ |\escapechar| is normally \TeX's internal code for
%backslash, but it can be changed if another convention is desired.
\ddanger 在到现在的例子中,|\string| 把控制系列转变为以 |\|$_{12}$ 开头的记号列表。%
但是这个反斜线记号没有真正连接到 \TeX\ 中;
有一个叫 |\escapechar| 的参数,它规定了控制系列输出为文本时所使用的字符。%
|\escapechar| 的值正常情况下是反斜线的 \TeX\ 内部代码,但是如果需要其它约定,
可以改变它。

%\ddanger \TeX\ has two other token-producing operations similar to the
%|\string| command. If you write ^|\number|\<number>, you get the decimal
%equivalent of the \<number>; and if you write ^|\romannumeral|\<number>,
%you get the number expressed in lowercase ^{roman numerals}. For example,
%`|\romannumeral24|' produces `|xxiv|', a list of four tokens each having
%category~12. The |\number| operation is redundant when it is applied
%to an explicit constant (e.g., `|\number24|' produces `|24|'); but it does
%suppress leading zeros, and it can also be used with numbers that are in
%\TeX's internal registers or parameters. For example, `|\number-0015|'
%produces `|-15|'; and if register |\count5| holds the value 316, then
%`|\number\count5|' produces `|316|'.
\ddanger 除了 |\string| 命令之外,\TeX\ 还有另外两个生成记号的命令。
如果你输入 |\number|\<number>,那么就得到等于 \<number> 的小数;
而如果你输入 |\romannumeral|\<number>,就得到用小写罗马数字表示的数。
\1例如,`|\romannumeral24|' 得到的是 `|xxiv|';
在这四个记号中,每个的类别码都是~12。
当用在显示常数时,|\number| 操作是多余的(比如,`|\number24|' 得到了 `|24|');
但是,它却去掉了开头的零,并且还可以用在 \TeX\ 内部寄存器或参数中的数字。%
例如,`|\number-0015|' 得到了 `|-15|';
并且如果寄存器 |\count5| 的值为 316,那么`|\number\count5|' 得到了 `|316|'。

%\ddanger The twin operations ^|\uppercase||{|\<token list>|}| and
%^|\lowercase||{|\<token list>|}| go through a given token list and convert
%all of the character tokens to their ``uppercase'' or ``lowercase''
%equivalents. Here's how: Each of the 256 possible characters
%has two associated values called the ^|\uccode| and the ^|\lccode|; these values
%are changeable just as a |\catcode| is. Conversion to uppercase means
%that a character is replaced by its |\uccode| value, unless the |\uccode|
%value is zero (when no change is made). Conversion to lowercase is
%similar, using the |\lccode|. The category codes aren't changed. When
%^|INITEX| begins, all |\uccode| and |\lccode| values are zero except that
%the ^{letters} |a| to~|z| and |A| to~|Z| have |\uccode| values |A| to~|Z|
%and |\lccode| values |a| to~|z|.
\ddanger 配对操作 |\uppercase||{|\<token list>|}| 和 |\lowercase||{|\<token list>|}|~%
处理一个给定的记号列表,并把它们的所有字符转换到相应的``大写''或``小写''字符。%
如下:256 个可能的字符都有两个相伴的\hbox{值,} 叫做 |\uccode| 和 |\lccode|;
这些值象 |\catcode| 一样是可以改变的。
转变到大写意味着用 |\uccode| 值来代替字符,否则 |\uccode| 的值就是零(当不改变时)。%
转变到小写类似,使用的是 |\lccode|。%
类代码不能改\hbox{变。}%
当 |INITEX| 运行时,所有 |\uccode| 和 |\lccode| 的值都是零,
除了字母 |a| 到 |z| 的 |\uccode| 为 |A| 到 |Z| 和 |A| 到 |Z| 的%
~|\lccode| 值为 |a| 到 |z|。

%\ddanger \TeX\ performs the  |\uppercase| and |\lowercase| transformations
%in its stomach, but the |\string| and |\number| and |\romannumeral|
%and |\csname| operations are carried out en route to the stomach (like
%macro expansion), as explained in Chapter~20.
\ddanger \TeX\ 在胃中完成 |\uppercase| 和 |\lowercase| 的变换,
但是 |\string|, |\number|, |\romannumeral| 和 |\csname| 操作在到胃的路上就完成了%
(就象宏的展开), 这些解释在第二十章。

%\ddangerexercise What token list results from
%`|\uppercase{a\lowercase{bC}}|'\thinspace?
%\answer First |\uppercase| produces `|A\lowercase{BC}|'; then you get `|Abc|'.
\ddangerexercise `|\uppercase{a\lowercase{bC}}|'得到是记号列表是什么?
\answer 首先 |\uppercase| 生成 `|A\lowercase{BC}|',然后你得到 `|Abc|'。

%\ddangerexercise \TeX\ has an internal integer parameter called ^|\year| that is
%set equal to the current year number at the beginning of every job. Explain how
%to use |\year|, together with |\romannumeral| and |\uppercase|, to
%print a copyright notice like \year=1986
%`\copyright\ \uppercase\expandafter{\romannumeral\year}'
%for all jobs run in \number\year.
%\answer `\thinspace|\copyright\ \uppercase\expandafter{\romannumeral\year}|%
%\thinspace'. \ (This is admittedly tricky; the `^|\expandafter|' expands
%the token after the `|{|', not the token after the group.)
\ddangerexercise \TeX\ 有一个内部整数参数叫做 |\year|,
它等于任务开始时当前年的数字。%
看看怎样用 |\year| 以及 |\romannumeral| 和 |\uppercase| 对所有运行在\number\year 的%
任务给出后面的版权表示:\year=1986
`\copyright\ \uppercase\expandafter{\romannumeral\year}'。
\answer `\thinspace|\copyright\ \uppercase\expandafter{\romannumeral\year}|\thinspace'。%
(这的确有些技巧性;`^|\expandafter|' 展开 `|{|' 后面的那个记号,而不是编组后面的那个记号。)

%\ddangerexercise Define a control sequence |\appendroman| with three parameters
%such that |\appendroman#1#2#3| defines control sequence |#1| to
%expand to a control sequence whose name is the name of control sequence
%|#2| followed by the value of the positive integer |#3| expressed in roman
%numerals.  For example, suppose |\count20| equals 30; then
%`|\appendroman\a\TeX{\count20}|' should have the same effect as
%`|\def\a{\TeXxxx}|'.^^{tricky macros}
%\answer (We assume that parameter |#2| is not simply an active character,
%and that ^|\escapechar| is between 0 and~255.)
%\begintt
%\def\gobble#1{} % remove one token
%\def\appendroman#1#2#3{\expandafter\def\expandafter#1\expandafter
%  {\csname\expandafter\gobble\string#2\romannumeral#3\endcsname}}
%\endtt
\ddangerexercise 定义三参数控制系列 |\appendroman|,使得 |\appendroman#1#2#3|
定义一个控制系列 |#1|,它展开后就是名字为整数参数为 |#3| 的控制系列 |#2| 的名字,
这个整数参数用罗马数字来表示。例如,假定 |\count20| 等于 30;
那么 `|\appendroman\a\TeX{\count20}|' 与 `|\def\a{\TeXxxx}|'有同样的效果。
\answer (我们假定参数 |#2| 不是一个活动符,且 ^|\escapechar| 在 0 和 255 之间。)
\begintt
\def\gobble#1{} % remove one token
\def\appendroman#1#2#3{\expandafter\def\expandafter#1\expandafter
  {\csname\expandafter\gobble\string#2\romannumeral#3\endcsname}}
\endtt

\endchapter

Some bookes are to bee tasted,
others to bee swallowed,
and some few to bee chewed and disgested.
\author FRANCIS ^{BACON}, {\sl Essayes\/} (1597) % p2 of orig edition

\bigskip

`Tis the good reader that makes the good book.
\author RALPH WALDO ^{EMERSON}, {\sl Society \& Solitude\/} (1870) % Success

\vfill\eject\byebye
